%%{ DOC HEAD

\pdfoutput=1
\documentclass[a4paper,11pt,titlepage,twoside]{book}

\usepackage[english]{babel}
\usepackage[utf8]{inputenc}
\usepackage{csquotes}

\usepackage{amsmath,amsfonts,amssymb,bm}
\usepackage{nicefrac}

\usepackage{algorithm,algpseudocode}
\usepackage[title,titletoc]{appendix}
\usepackage{latexsym}
\usepackage{a4wide}
\usepackage{color}
\usepackage{indentfirst}
\usepackage{graphicx}       %%% graphics for dvips
\usepackage{fancyhdr,lastpage}
\usepackage{longtable}
\usepackage{pifont}
\usepackage{makeidx}
\usepackage{multirow}
\usepackage{dcolumn}
\usepackage{epstopdf}
\usepackage{url}
\usepackage{listings}
\usepackage{relsize}
\usepackage{pdfpages}
\usepackage{url}
\usepackage{lipsum}
\usepackage{isotope}
\usepackage{verbatim}
\usepackage{xcolor}
\usepackage{tcolorbox}
\usepackage[hidelinks]{hyperref}

\usepackage{subfig} % subfloat
\usepackage[export]{adjustbox}

\hyphenation{}

%%{ tikz

\usepackage{tikz}
\usepackage{pgfplots}
\pgfplotsset{compat=1.14}
\usetikzlibrary{backgrounds,arrows,automata,shapes,positioning,calc,through,spy,shapes,shapes.geometric,shapes.multipart,fit,patterns,fadings}
\pgfdeclarelayer{background}
\pgfdeclarelayer{foreground}
\pgfsetlayers{background,main,foreground}

\tikzset{
    imgletter/.style={
      rectangle,
      inner sep=2pt,
      rounded corners=.1em,
      text=black,
      minimum height=1em,
      text centered,
      fill=white,
      fill opacity=1.0,
      text opacity=1,
      anchor=south west,
  },
}

%%}

%%{ include paper box

% \newcommand{\includepaper}[1]{\conditionalClearPage \fullciteinbox{#1}{This place will contain the full paper}}

\newcommand{\includepaperwithimage}[1]{
  \conditionalClearPage
  \fullciteinboxwithimage{#1}{This place will contain the full paper}
}

% \newcommand{\includepaperwithimage}[1]{\includepdf[scale=0.85,pages=-,pagecommand={\thispagestyle{plain}}]{./papers_to_include_pdf/#1}}

% \newcommand{\includepaper}[1]{\fullciteinbox{#1}{This place will contain the full paper}}

%%}

%%{ fullcite box

\definecolor{light-gray}{gray}{0.95}
\newcommand{\fullciteinbox}[2]{%

\DeclareCiteCommand{\fullcite}
{\usebibmacro{prenote}}
{\clearfield{addendum}%
  \usedriver
  {\defcounter{minnames}{6}%
  \defcounter{maxnames}{6}}
{\thefield{entrytype}}}
{\multicitedelim}
{\usebibmacro{postnote}}

%\vspace{3em}%
%\raisebox{3em}[3em][3em]{%
% \vspace{-0.2em}
\begin{tcolorbox}[width=\textwidth,colback={light-gray},title={}]%
\ifx&#2&
\else
  \textbf{#2}:\\\\
\fi
\begin{minipage}[t]{0.07\linewidth}%
\raggedright%
\cite{#1}%
\end{minipage}%
\begin{minipage}[t]{0.93\linewidth}%
\fullcite{#1}%
\end{minipage}%
\end{tcolorbox}%
%}%
% \vspace{-0.5em}
}%

%%}

%%{ fullcite box with image

\definecolor{light-gray}{gray}{0.95}
\newcommand{\fullciteinboxwithimage}[2]{%

\DeclareCiteCommand{\fullcite}
{\usebibmacro{prenote}}
{\clearfield{addendum}%
  \usedriver
  {\defcounter{minnames}{6}%
  \defcounter{maxnames}{6}}
{\thefield{entrytype}}}
{\multicitedelim}
{\usebibmacro{postnote}}

%\vspace{3em}%
%\raisebox{3em}[3em][3em]{%
% \vspace{-0.2em}
\begin{tcolorbox}[width=\textwidth,colback={light-gray},title={}]%
\ifx&#2&
\else
  \textbf{#2}:\\\\
\fi
\begin{minipage}[t]{0.07\linewidth}%
\raggedright%
\cite{#1}%
\end{minipage}%
\begin{minipage}[t]{0.93\linewidth}%
\fullcite{#1}
\end{minipage}%
\end{tcolorbox}
\begin{center}
  \tikzfading[name=fade down,
    top color=transparent!100, bottom color=transparent!0]
  \begin{tikzpicture}
    \node[anchor=south west,inner sep=0] (a) at (0,0) {\includegraphics[height=0.8\textheight]{./papers_to_include_pdf/#1.png}};
    \begin{scope}[x={(a.south east)},y={(a.north west)}]
      \fill[white, path fading=fade down] (0.0, 0.0) rectangle (1.0, 0.33);
    \end{scope}
  \end{tikzpicture}
\end{center}
%}%
% \vspace{-0.5em}
}%

%%}

%%{ acronym

\usepackage[nolist,nohyperlinks]{acronym}
\acrodef{GPS}[GPS]{Global Positioning System}
\acrodef{SLAM}[SLAM]{Simultaneous Localization And Mapping}
\acrodef{SLAMs}[SLAMs]{Simultaneous Localization And Mapping systems}
\acrodef{GPS}[GPS]{Global Positioning System}
\acrodef{RTK}[RTK]{Real-time Kinematics}
\acrodef{GNSS}[GNSS]{Global Navigation Satellite System}
\acrodef{ROS}[ROS]{Robot Operating System}
\acrodef{API}[API]{Application Programming Interface}
\acrodef{UAV}[UAV]{Unmanned Aerial Vehicle}
\acrodef{MAV}[MAV]{Micro Aerial Vehicle}
\acrodef{UGV}[UGV]{Unmanned Ground Vehicle}
\acrodef{UV}[UV]{Ultra-Violet}
\acrodef{LED}[LED]{Light-emitting Diode}
\acrodef{MBZIRC}[MBZIRC]{Mohamed Bin Zayed International Robotics Challenge}
\acrodef{DARPA}[DARPA]{Defense Advanced Research Projects Agency}
\acrodef{IMU}[IMU]{Inertial Measurement Unit}
\acrodef{LTI}[LTI]{Linear time-invariant}
\acrodef{MPC}[MPC]{Model Predictive Control}
\acrodef{UVDAR}[UVDAR]{Ultra-Violet Direction And Ranging}
\acrodef{DOF}[DOF]{degree-of-freedom}
\acrodef{DOFs}[DOFs]{degrees-of-freedom}
\acrodef{LiDAR}[LiDAR]{Light Detection and Ranging}
\acrodef{ESC}[ESC]{Electronic Speed Controller}
\acrodef{LKF}[LKF]{Linear Kalman Filter}
\acrodef{UKF}[UKF]{Unscented Kalman Filter}
\acrodef{EKF}[EKF]{Extended Kalman Filter}
\acrodef{RAS}[RAS]{Robotics and Automation Society}
\acrodef{IEEE}[IEEE]{Institute of Electrical and Electronics Engineers}
\acrodef{MRS}[MRS]{Multi-robot Systems Group}
\acrodef{FOV}[FOV]{Field of View}
\acrodef{CdTe}[CdTe]{Cadmium Telluride}
\acrodef{FDNPP}[FDNPP]{Fukushima Daiichi Nuclear Power Plant}
\acrodef{CTU}[CTU]{Czech Technical University}
\acrodef{ISS}[ISS]{International Space Station}
\acrodef{ASIC}[ASIC]{Application-Specific Integrated Circuit}
\acrodef{CMOS}[CMOS]{Complementary Metal Oxide Semiconductor}
\acrodef{CCD}[CCD]{Charge-Coupled Device}
\acrodef{LEO}[LEO]{Low-Earth Orbit}

%%}

%%{ siunitx

\usepackage{siunitx}
\DeclareSIUnit \parsec {pc}
\DeclareSIUnit \electronvolt {eV}
\DeclareSIUnit \pixel {px}
\DeclareSIUnit \arcmin {arcmin}
\DeclareSIUnit \erg {erg}
\DeclareSIUnit \joul {J}

%%}

%%{ change formatting of lists

\usepackage{enumitem}
\setlist{nosep}
% \setlist{noitemsep}
% how to format particular lists?
% \begin{itemize}[topsep=8pt,itemsep=4pt,partopsep=4pt, parsep=4pt]

%%}

%%{ change spacing of the table of contents

% \usepackage{tocloft}
% \renewcommand\cftchapafterpnum{\vskip5pt}
% \renewcommand\cftsecafterpnum{\vskip5pt}

%%}

%%{ change formatting of a chapter

\usepackage{titlesec}
\titleformat{\chapter}[block]
{\normalfont\huge\bfseries}{Chapter \thechapter\\\vspace{0.1em}\\}{1em}{\Huge}
% {?}{before}{after}
\titlespacing*{\chapter}{0pt}{-1em}{2em}

%%}

%%{ biblatex

\usepackage[backend=bibtex,defernumbers=true,style=ieee,sorting=ydnt,sortcites=true]{biblatex}

\renewcommand*{\bibfont}{\Font}

% \newcounter{mycounter}
% \setcounter{mycounter}{0}
% \newcounter{unrelatedcount}
% \setcounter{unrelatedcount}{0}
% \newcounter{totalcounter}
% \setcounter{totalcounter}{0}

% % Print labelnumbers with suffixes, adjust secondary labelnumber 1/2 (start new numbering of my publications)
% \makeatletter
% \AtDataInput{%
%   \ifkeyword{mine}
%   {
%     \addtocounter{mycounter}{1}
%   }
%   {}
%   \addtocounter{totalcounter}{1}
% }
% \makeatother

% Print labelnumbers with suffixes, adjust secondary labelnumber 2/2
\DeclareFieldFormat{labelnumber}{%
  \ifkeyword{mine}
    {\ifkeyword{core}
      {{\number\numexpr#1}c}%
      {{\number\numexpr#1}a}%
    }%
    {#1}%
}

\DeclareCiteCommand{\tabcite}%[\mkbibbrackets]
  {\usebibmacro{cite:init}%
   \usebibmacro{prenote}}
  {\usebibmacro{citeindex}%
   \usebibmacro{cite:comp}}
  {}
  {\usebibmacro{cite:dump}%
   \usebibmacro{postnote}}

% {{\number\numexpr#1-\value{bbx:primcount}}a}

\addbibresource{main.bib}

\defbibenvironment{favoritebib}
{\itemize}
{\enditemize}
{\item}

%%}

%%{ custom macros

%%%%%%%%%%%%%%% changemargin environment begin %%%%%%%%%%%%%%%%%%%%%%%%%
\def\changemargin#1#2{\list{}{\rightmargin#2\leftmargin#1}\item[]}
\let\endchangemargin=\endlist
%%%%%%%%%%%%%%% changemargin environment end %%%%%%%%%%%%%%%%%%%%%%%%%

\newcommand{\strong}[1]{\textbf{#1}}
\newcommand{\coord}[1]{\textbf{#1}}
\newcommand{\norm}[1]{\left\lvert#1\right\rvert}
\newcommand{\m}[1]{\ensuremath{\mathbf{#1}}}
\newcommand\numberthis{\addtocounter{equation}{1}\tag{\theequation}}
\newcommand{\corrected}[1]{{\color{black} {#1}}}
% \newcommand{\comment}[1]{{\color{blue} {#1}}}
\newcommand{\add}[1]{{\color{green} {#1}}}
\newcommand{\todo}[1]{{\color{red} TODO {#1}}}
\newcommand{\updated}[1]{{\color{blue} {#1}}}
\newcommand{\reffig}[1]{Fig.~\ref{#1}}
\newcommand{\refalg}[1]{Alg.~\ref{#1}}
\newcommand{\refsec}[1]{Sec.~\ref{#1}}
\newcommand{\reftab}[1]{Table~\ref{#1}}
\newcommand{\real}{\mathbb{R}}
\newcommand{\red}[1]{{\color{red} #1}}
\newcommand{\minus}{\scalebox{0.75}[1.0]{$-$}}
\newcommand{\plus}{\scalebox{0.8}[0.8]{$+$}}
\newcommand{\figvspace}{\vspace{-1em}} % this may eventually do something, so far just a placeholder

\newcommand{\chapternoclear}[1]{
  \begingroup
  \let\cleardoublepage\clearpage
  \chapter{#1}
  \endgroup
}

\newcommand{\conditionalClearPage}{
  \ifdefined\printversion
  %\newpage{}
  %\thispagestyle{empty}
  \clearemptydoublepage
  %\cleardoublepage{\thispagestyle{empty}}
  \else
  \newpage{}
  \clearpage
  \fi
}

%%}

%%{ listings

\lstset{breaklines=true,captionpos=b,frame=single,language=sh,float=h}
\lstloadlanguages{sh,c}
\def\lstlistingname{Listing}
\def\lstlistlistingname{Listings}

%%}

%%{ title page parameters

\newcommand{\Author}{Ing. Tomáš Báča}
\newcommand{\Supervisor}{Ing. Martin Saska, Dr. rer. nat.}
\newcommand{\SupervisorSpecialist}{Ing. Michal Platkevic, Ph.D.}
\newcommand{\Programme}{Electrical Engineering and Information Technology}
\newcommand{\Field}{Artificial Intelligence and Biocybernetics}
\newcommand{\Title}{Cooperative Sensing by a Group\\[0.5em]of Unmanned Aerial Vehicles}
\newcommand{\DocName}{Doctoral Thesis}
\newcommand{\Keywords}{Unmanned Aerial Vehicles, Ionizing localization}
\newcommand{\DOCVersion}{0.1}
\newcommand{\Year}{2020}
\newcommand{\Month}{September}
\newcommand{\Date}{\Month~\Year}
\newcommand{\Location}{Prague}

%%}

%%{ layout parameters

% % altering margins
% \setlength{\oddsidemargin}{+0.5cm}
% \setlength{\evensidemargin}{-0.5cm}

% ??
\def\clinks{false}

% no indent, free space between paragraphs
\setlength{\parindent}{1cm}
\setlength{\parskip}{1ex plus 0.5ex minus 0.2ex}

% offsets the head down
\setlength{\headheight}{18pt}

% foot line
\renewcommand{\footrulewidth}{0.4pt}

\fancypagestyle{plain}

% clear the default layout
\fancyhead{}
\fancyfoot{}

% page header
\fancyhead[LO]{\leftmark}
\fancyhead[RE]{\rightmark}
\fancyhead[LE,RO]{\thepage/\pageref{LastPage}}

% page footer
\fancyfoot[L]{CTU in Prague}
\fancyfoot[R]{Department of Cybernetics}
\fancyfoot[C]{}

%%}

%%{ other parameters

% European layout (no extra space after `.')
\frenchspacing

% without this it does not compile!
\let\bibfont\small

%%}

%%}

\begin{document}

\begin{titlepage}
  \begin{center}

    \textsc{\Large Czech Technical University in Prague}\\[1em]
    \textsc{\large Faculty of Electrical Engineering\\
    Department of Cybernetics\\
    Multi-robot Systems\\[3em]
    }
    \includegraphics[height=4.1cm]{fig/lev.pdf}\\[3em]

    \textbf{\textsc{\Huge \Title}}\\[2em]

    \textbf{\Large \DocName}\\[6em]

    \textbf{\huge \Author}\\[6em]

    {\large \Location, \Date}\\[3em]

    Ph.D. programme: Electrical Engineering and Information Technology\\
    Branch of study: Artificial Intelligence and Biocybernetics\\[2em]

    \textbf{Supervisor: \Supervisor}\\
    \textbf{Supervisor-Specialist: \SupervisorSpecialist}

    \vspace{2pt}

  \end{center}
\end{titlepage}


\conditionalClearPage
%!TEX root = ../main.tex

~\vfill{}

\section*{Acknowledgments}

\todo{}

During my Ph.D. studies, I have been supported by the Czech Technical University in Prague through a Ph.D. scholarship and by the grants SGS15/157/OHK3/2T/13 and SGS17/187/OHK3/3T/13.
The Ministry of education of the Czech Republic supported the work by the grant no. 7AMB16FR017, and no. LH11053, and by OP VVV funded project CZ.02.1.01/0.0/0.0/16\_019/0000765 ``Research Center for Informatics''.
The Czech Science Foundation supported this work through projects no. 17-16900Y, no. 18-10088Y, and no. 20-10280S.
The Technology Agency of the Czech Republic supported this work through project no. FW01010317.
The European Union’s Horizon 2020 research and innovation programme supported this work under the grant agreement No 871479.
The National Grid Infrastructure MetaCentrum provided access to computing and storage facilities under the programme CESNET 569/2015, and LM2015042.
The Khalifa University of Science funded our participation in the MBZIRC 2017 and MBZIRC 2020 competitions that also motivated this work.
The work on the outer space radiation dosimetry and measurements would not be possible without the support of the Technology Agency of the Czech Republic projects no. TA03011329, no. TA04011295, the Czech Science Foundation projects no. GA13-33324S, GJ18-10088Y, and the project MSMT LH14039 of the Ministry of education youth and sports of the Czech Republic.
The work has been done on behalf of Medipix2 and Medipix3 collaborations.

\vspace{2.5cm}


\conditionalClearPage
%!TEX root = ../main.tex
\begin{changemargin}{0.8cm}{0.8cm} 

~\vfill{}

\section*{Copyright}
\vskip 0.5em

\todo{}
This thesis is a compilation of several journal articles published during my PhD studies.
The included publications are presented in accordance with the copyrights of IEEE, Springer Nature, Elsevier, and Wiley for posting the works for internal institutional uses.
The works are protected by the copyrights of respective publishers and can not be further reprinted without the permission of the publishers. 

\vskip 2.5cm

\textsuperscript{\textcopyright} \todo{}\\
\end{changemargin} 


\conditionalClearPage
\input{src/abstract.tex}

\pagestyle{fancy}

\conditionalClearPage
\tableofcontents

%% | ------------------------ Chapter 1 ----------------------- |

%%{ Introduction

\chapternoclear{Introduction}

The emergence of small \acp{UAV} has created a new active field of mobile robotics.
The rise of multirotor helicopters spawned revolutionary possibilities of remote sensing and data gathering.
Unlike traditional ground robots, multirotor helicopters combine a potentially fast and agile movement through a 3D environment with the ability to hover in place.
Both traits require overcoming complex technical challenges as well as offer significant advantages over ground robots.
The challenges arise from the inherently unstable dynamics of multirotor helicopters \cite{kumar2012opportunities, mueller2014stability}.
Uninterrupted feedback control actions are required to maintain the machine in the desired state.
Moreover, feedback control of agile \acp{UAV} is vitally dependent on a smooth and feasible state estimate.
Both the \ac{UAV} state estimation \cite{merino2006vision, burri2015robust, grabe2015nonlinear} and feedback control \cite{lee2010geometric, goodarzi2015geometric, kamel2017robust} have been intensively studied during the past decade and are still a very active field of research.
Multirotor helicopters are a versatile platform for carrying out remote sensing \cite{colomina2014unmanned, pajares2015overview}, environmental sampling, and providing technical support and aid in natural disaster rescue operations \cite{yuan2015survey, perks2016advances}.

Remote sensing can be traditionally performed by a stationary sensory system, a satellite, or a remotely controlled robot.
In recent years, autonomous robotic remote sensing became available with the emergence of autonomous systems.
Onboard autonomy is used to control a robotics system when employing a human operator might not be possible, e.g., deep underground \cite{apachristos2019autonomous, losch2018design}, or in a vicinity of a damaged nuclear power plant \cite{sato2019radiation}.
Multi-robotic distributed sensing is, however, still in its infancy.
Although distributing the process offers robustness through redundancy and a potential increase in information yield, it also poses new challenges in multi-robotic coordination and sensor fusion.
The same applies to the use of multiple multirotor \ac{UAV} with even more challenges.
Just on its own, control of multi-\ac{UAV} swarms and formations is a challenging subfield which has direct implications to the field of distributed remote sensing.

One of the most iconic sub-fields of remote sensing is the remote sensing of ionizing radiation.
Due to the unusual nature of ionizing radiation measured \cite{andreo2017fundamentals}, but also due to the inherent danger ionizing radiation poses to living organisms.
Recent advances in semiconductor technologies allow the fabrication of small semiconductor radiation detectors, which opened up a possibility to measure ionizing radiation onboard small \acp{UAV}.
Thanks to onboard autonomy, small \acp{UAV} can be deployed to map ionizing radiation and even localize radioactive sources autonomously.

This thesis focuses on advances in replicable research with autonomous multi-rotor helicopters and their use for remote sensing and remote sensing of ionizing radiation.
The thesis's objectives were significantly shaped by the active participation of the \ac{MRS} research group in the 2017 and 2020 rounds of the \ac{MBZIRC}.
MBZIRC proposed a set of robotics challenges that pushed the boundaries in \ac{UAV} autonomy.
The challenges ranged from an autonomous landing on a moving car by an unmanned aircraft, collaborative gathering of small objects by a fleet of \acp{UAV}, up to an autonomous brick wall-building by a fleet of \acp{UAV}.
Those tasks provided excellent opportunity and conditions to take state-of-the art techniques outside of lab and compare the advances directly to the best university teams from around the world.

Furthermore, interdisciplinary research is being pursued to localize ionizing radiation sources by \acp{UAV}.
The research closely follows advances in the use of novel imaging pixel radiation detectors, initially developed for medical imaging, in the remote sensing application of space dosimetry and X-ray imaging.
Initial steps has been made by the thesis author to develop a remote sensing module with the Timepix~\cite{llopart2007timepix} detector for the first Czech CubeSat, the VZLUSAT-1.
One of the goals of the thesis is to transfer the know how to the field of small \aclp{UAV}.

Experimental verification of novel methods is crucial to provide objective evaluation and to support new publications.
Although, the accompanying sub-fields of cybernetics --- machine learning and computer vision --- benefit significantly from evaluation and comparison of new methods on datasets, this does not apply to mobile robotics.
Even realistic simulation do not fully substitute the testing and verification using a real \ac{UAV} equipped with real sensors, and most importantly, outside of laboratory conditions.
However, conducting real-world experiments requires an onboard control platform that can satisfy the needs of the tested method.
Despite the plethora of existing solutions \cite{sanchez2016aerostack, xiao2020xtdrone, furrer2016rotors, schmittle2018openuav, abeywardena2015design, mellado2013mavwork}, none provides all the features needed to support the work presented in the thesis.
To name a few, not all platforms support experiments both indoors and outdoors.
The \ac{UAV} state estimation is often limited to a single localization approach.
Mid-air switching of controllers and control reference generators is also not common.
Finally, low-level control output using a desired \ac{UAV} attitude rate is rarely present.

%%{ Fig: intro collage

\begin{figure}[!htb]
  \centering
  \subfloat {
    \includegraphics[width=0.30\textwidth]{./fig/photos/control_2_1-5.jpg}
  }
  \subfloat {
    \includegraphics[width=0.30\textwidth]{./fig/photos/swarm_2_1-5.jpg}
  }
  \subfloat {
    \includegraphics[width=0.30\textwidth]{./fig/photos/grasping_2017_2_1-5.jpg}
  }\\
  \vspace{-0.3em}
  \subfloat {
    \includegraphics[width=0.30\textwidth]{./fig/photos/brick_placing_2_1-5.jpg}
  }
  \subfloat {
    \includegraphics[width=0.30\textwidth]{./fig/photos/landing_2017_2_1-5.jpg}
  }
  \subfloat {
    \includegraphics[width=0.30\textwidth]{./fig/photos/darpa_drone_beta_2_1-5.jpg}
  }\\
  \vspace{-0.3em}
  \subfloat {
    \includegraphics[width=0.30\textwidth]{./fig/photos/radron_vio_2_1-5.jpg}
  }
  \subfloat {
    \includegraphics[width=0.30\textwidth]{./fig/photos/vzlusat_1_1-5.jpg}
  }
  \subfloat {
    \includegraphics[width=0.30\textwidth]{./fig/photos/rex_1_1-5.jpg}
  }
  \caption{Illustration of outcomes of the thesis author's contributions: deployments of multi-rotor \acp{UAV} as well as radiation measurements for space applications.}
  \label{fig:collage}
\end{figure}

%%}

The objectives of the thesis are summarized as follows:

\textbf{(1) Development of a control system for real-world deployment of \acp{UAV}}, verification of new methods for control, remote sensing, and deployment in indoor and outdoor environments.
Despite many platforms for multirotor \acp{UAV} control and deployment are available, they lack features necessary for real-world testing and deployment of the methods within the focus of this thesis and the focus of the \acl{MRS} group at \ac{CTU} in Prague.
Therefore, the first objective and a long-term effort of the author are to develop a modular \ac{UAV} control system.
The control system should allow safe indoor and outdoor deployment of multirotor helicopters, allow verification of high-level methods for motion planning, multi-\ac{UAV} swarming, and formation flying.
The system should also allow basic research on low-level control and stabilization of the multirotor \ac{UAV} dynamics.

\textbf{(2) Research of methods of collaborative remote sensing} by a group of \aclp{UAV} in real-world non-laboratory conditions.
Collaboratively executing UAV missions poses challenges on many onboard autonomy levels, e.g., task allocation, estimation, motion planning, and control.
Furthermore, mutual communication between the \acp{UAV} might be unreliable or completely unavailable.
Therefore, sharing real-time sensor readings to pursue a common goal might not be possible in all circumstances.
Moreover, mutual collisions between the \acp{UAV} can be expected if the \acp{UAV} are guided by common goals.
The objective is to explore and push the field of collaborative remote sensing and deployment of \ac{UAV} in complex robotics tasks forwards.

\textbf{(3) Advancing the field of ionizing radiation dosimetry, mapping, and localization of compact sources} by \aclp{UAV}.
Ionizing radiation has been traditionally measured onboard \acp{UAV} using dosimeters \cite{nagatani2013emergency, sanada2015aerial, towler2012radiation, jiang2016prototype} --- sensors measuring the intensity of incoming radiation.
Often, the intensity is utilized only to estimate the scalar field of radiation intensity.
Rarely, direction measurement can be obtained with an additional device, e.g., the optical collimator or a coded aperture.
However, those solutions are not well-suited for small \acp{UAV} due to the heavyweight of the sensor equipment.
This thesis aims to push the state-of-the-art by utilizing miniature semi-conductor pixel detectors \cite{llopart2007timepix} and novel event-based radiation detectors \cite{poikela2014timepix3} onboard miniature \acp{UAV}.

The rest of the thesis is organized as follows.
This thesis is a compilation of 8 included core publications, referenced as 1c -- 8c.
Furthermore, the thesis is supported by additional related authored publications, referenced as 9c -- 42c.
Firstly, state of the art is summarized in Chapter~\ref{chap:sota}.
Chapter~\ref{chap:uav_platform} introduces the publications related to the developed \ac{UAV} platform.
Chapter~\ref{chap:sensing} covers the publications related to the multi-\ac{UAV} sensing and deployment.
Finally, Chapter~\ref{chap:radiation} presents publications on radiation measurement, localization, and mapping.

%%}

%% | ------------------------ Chapter 2 ----------------------- |

%%{ Contributions and Related Work

\chapternoclear{Contributions and Related Work\label{chap:sota}}

\section{Author's publications and contributions}

Figure~\ref{fig:research_graph} shows a publication graph composed of accepted peer-review publications and publications that have been submitted in the time of writing this thesis.
The publications are split into four main categories.
The first category, a \emph{Pre-Ph.D. research}, consists of publications based upon the research done before the author started the pursue of Ph.D. ($\leq 2015$).
Although, some of these were written and submitted during the author's Ph.D. studies.
The second category is the research stream \ac{UAV} platform development for research validation and reliable deployment of novel methods in control, navigation, formation flying, and swarming.
Thirdly, the largest group of the author's publications is from the field of multi-\ac{UAV} remote sensing, swarming, and deployment.
Lastly, the fourth category is researching ionizing radiation detection, mapping and localization.
The author conducts an interdisciplinary transfer from space-oriented physics field to the field of unmanned vehicles such as multirotor helicopters.

%%{ Fig: research graph

\begin{figure}

  %%{ tikzset

  \tikzset{
    article/.style={
      rectangle,
      inner sep=0pt,
      draw=black,
      text=black,
      minimum height=3.5em,
      minimum width=6.7em,
      text width=6.7em,
    },
    related/.style={
      rectangle,
      dashed,
      inner sep=0pt,
      draw=black,
      text=black,
      minimum height=3.5em,
      minimum width=6.7em,
      text width=6.7em,
      text height=3.0em,
      align=center,
    },
    subarticle/.style={
      rectangle,
      inner sep=0pt,
      draw=black,
      text=black,
      fill=black!15,
      minimum height=3.5em,
      minimum width=6.7em,
      text width=6.7em,
    },
    subkeyarticle/.style={
      rectangle,
      inner sep=0pt,
      draw=black,
      line width=0.5mm,
      text=black,
      fill=black!15,
      minimum height=3.5em,
      minimum width=6.7em,
      text width=6.7em,
    },
    keyarticle/.style={
      rectangle,
      inner sep=2pt,
      draw=black,
      text=black,
      line width=0.5mm,
      minimum height=3.5em,
      minimum width=6.5em,
      text width=6.5em,
    },
    stream/.style={
      rectangle,
      inner sep=2pt,
      draw=black,
      text=black,
      line width=0.5mm,
      minimum height=3em,
      minimum width=8em,
      rounded corners=1em,
      font=\bfseries,
    },
    technology/.style={
      rectangle,
      inner sep=2pt,
      draw=black,
      text=black,
      line width=0.5mm,
      minimum height=3.5em,
      minimum width=6.5em,
      rounded corners=0.5em,
      align=left,
    },
    cite/.style={
      rectangle,
      inner sep=2pt,
      text=black,
      draw=black,
      % minimum height=1em,
      text centered,
      anchor=south east,
    },
    keycite/.style={
      rectangle,
      inner sep=2pt,
      text=black,
      draw=black,
      line width=0.5mm,
      % minimum height=1em,
      text centered,
      anchor=south east,
    },
  }

  %%}

  %%{ commands

  \newcommand{\stream}[3]{
    \node[stream, #1] (#2) {\scriptsize #3};
  }

  \newcommand{\technology}[3]{
    \node[technology, #1] (#2) {\tiny #3};
  }

  \newcommand{\keyarticle}[4]{
    \node[keyarticle, #1] (#2) {\scalebox{0.9}{\tiny #4}};
    \node[keycite] at (#2.south east) {\tiny #3};
  }

  \newcommand{\article}[4]{
    \node[article, #1] (#2) {\scalebox{0.9}{\tiny #4}\vspace{0.1em}};
    \node[cite] at (#2.south east) {\tiny #3};
  }

  \newcommand{\related}[3]{
    \node[related, #1] (#2) {\scalebox{0.9}{\tiny #3}};
  }

  \newcommand{\subarticle}[4]{
    \node[subarticle, #1] (#2) {\scalebox{0.9}{\tiny #4}\vspace{0.1em}};
    \node[cite] at (#2.south east) {\tiny #3};
  }

  \newcommand{\subkeyarticle}[4]{
    \node[subkeyarticle, #1] (#2) {\scalebox{0.9}{\tiny #4}\vspace{0.1em}};
    \node[keycite] at (#2.south east) {\tiny #3};
  }

  %%}

  \centering

  \begin{adjustbox}{width=1.00\textwidth}

  \begin{tikzpicture}[node distance=1.0em and 1.0em, auto]

    %%{ block type headlines

    \stream{}{stream_generic}{Research stream}

    \technology{right=of stream_generic}{technology_generic}{
      \begin{tabular}{l}
        Key technology\\
        developped by\\
        the thesis author
      \end{tabular}
    }

    \keyarticle{right=of technology_generic}{keyarticle_generic}{1c}{
      \begin{tabular}{l}
        Key article\\
        included in\\
        the thesis
      \end{tabular}
    }

    \article{right=of keyarticle_generic}{article_generic}{1a}{
      \begin{tabular}{l}
        Authored article
      \end{tabular}
    }

    \subarticle{right=of article_generic}{subarticle_generic}{2a}{
      \begin{tabular}{l}
        Submitted article
      \end{tabular}
    }

    \related{right=of subarticle_generic}{dashedbox_generic}{
      Related articles
    }

    %%}

    %%{ precursor publications

    \article{below=of stream_generic,shift={(2.0em, -2em)}}{saska2013adhoc}{\tabcite{saska2013adhoc}}{
      \begin{tabular}{l}
        UAV-UGV\\
        formations unde\\
        relative localization\\
        \emph{IROS 2013}
      \end{tabular}
    }

    \article{right=of saska2013adhoc}{chudoba2014localization}{\tabcite{chudoba2014localization}}{
      \begin{tabular}{l}
        UAV localization\\
        and stabilization\\
        using visual features\\
        \emph{ICUAS 2014}
      \end{tabular}
    }

    \article{right=of chudoba2014localization}{baca2016embedded}{\tabcite{baca2016embedded}}{
      \begin{tabular}{l}
        Embedded MPC\\
        for UAV control\\
        \emph{MMAR 2016}
      \end{tabular}
    }

    \article{right=of baca2016embedded}{saska2017documentation}{\tabcite{saska2017documentation}}{
      \begin{tabular}{l}
        Documentation of\\
        historical buildings\\
        using UAVs\\
        \emph{ETFA 2017}
      \end{tabular}
    }

    \article{below=of saska2013adhoc,shift={(0em, 0em)}}{chudoba2016exploration}{\tabcite{chudoba2016exploration}}{
      \begin{tabular}{l}
        Exploration for\\
        visual feature-based\\
        UAV navigation\\
        \emph{JIRS 2016}
      \end{tabular}
    }

    \article{right=of chudoba2016exploration}{saska2016formations}{\tabcite{saska2016formations}}{
      \begin{tabular}{l}
        UAV formations\\
        with migratin\\
        virtual leader\\
        \emph{ICARCV 2016}
      \end{tabular}
    }

    \article{right=of saska2016formations}{spurny2016complex}{\tabcite{spurny2016complex}}{
      \begin{tabular}{l}
        Complex manouvres\\
        of UAV-UGV\\
        formations\\
        \emph{MMAR 2016}
      \end{tabular}
    }

    \article{right=of spurny2016complex}{saska2017system}{\tabcite{saska2017system}}{
      \begin{tabular}{l}
        System for UAV\\
        in GPS-denied\\
        environments\\
        \emph{AuRo 2017}
      \end{tabular}
    }

    \article{right=of baca2016embedded, shift={(10.5em, 0.0em)}}{baca2016miniaturized}{\tabcite{baca2016miniaturized}}{
      \begin{tabular}{l}
        X-Ray telescope\\
        with Timepix sensor\\
        for VZLUSAT-1\\
        nanosatellite\\
        \emph{JINST 2016}
      \end{tabular}
    }

    \article{below=of baca2016miniaturized}{daniel2016terrestrial}{\tabcite{daniel2016terrestrial}}{
      \begin{tabular}{l}
        Terrestrial gamma\\
        ray monitor on\\
        a cubesat\\
        \emph{SPIE O\&P 2016}\\
      \end{tabular}
    }

    %%}

    %%{ research stream headlines

    \stream{below=of stream_generic, shift={(0.00em, -11.5em)}}{uav_platform_stream}{
      \begin{tabular}{l}
        UAV platform\\
        for research
      \end{tabular}
    }

    \stream{right=of uav_platform_stream}{uav_sensing_stream}{
      \hspace{4em}
      \begin{tabular}{l}
        Application of UAV in remote\\
        sensing and deployment
      \end{tabular}
      \hspace{4em}
    }

    \stream{right=of uav_sensing_stream}{radiation_stream}{
      \hspace{3em}
      \begin{tabular}{l}
        Ionizing Radiation\\
        dosimetry and imaging
      \end{tabular}
      \hspace{3em}
    }

    \node [rotate=90, left=of stream_generic, shift={(-4.2em, -1.5em)}] {Pre-Ph.D. research};

    %%}

    %%{ blocks

    \keyarticle{below=of radiation_stream,shift={(-4em, 0)}}{baca2018timepix}{\tabcite{baca2018timepix}}{
      \begin{tabular}{l}
        1 year of\\
        VZLUSAT-1\\
        dosimetry in orbit\\
        \emph{JINST 2018}
      \end{tabular}
    }

    \article{right=of baca2018timepix}{urban2017vzlusat}{\tabcite{urban2017vzlusat}}{
      \begin{tabular}{l}
        VZLUSAT-1\\
        nanosatellite\\
        \emph{AA 2017}
      \end{tabular}
    }

    \article{below=of urban2017vzlusat}{daniel2019inorbit}{\tabcite{daniel2019inorbit}}{
      \begin{tabular}{l}
        Commissioning\\
        of VZLUSAT-1\\
        nanosatellite\\
        \emph{SSR 2019}
      \end{tabular}
    }

    \technology{below=of baca2018timepix}{rospix_technology}{
      \begin{tabular}{l}
        Rospix: Timepix\\
        controller\\
        for ROS
      \end{tabular}
    }

    \technology{below=of uav_platform_stream}{mpc_tracker_technology}{
      \begin{tabular}{l}
        MPC Tracker for\\
        UAV trajectory\\
        following
      \end{tabular}
    }

    \keyarticle{below=of uav_sensing_stream,shift={(-4em, 0em)}}{loianno2018localization}{\tabcite{loianno2018localization}}{
      \begin{tabular}{l}
        Localization of\\
        objects by UAVs\\
        \emph{RA-L 2018}
      \end{tabular}
    }

    \keyarticle{right=of loianno2018localization}{spurny2019cooperative}{\tabcite{spurny2019cooperative}}{
      \begin{tabular}{l}
        Cooperative object\\
        gathering by UAVs\\
        \emph{JFR 2019}
      \end{tabular}
    }

    \article{below=of loianno2018localization,shift={(0em, 0em)}}{baca2017autonomous}{\tabcite{baca2017autonomous}}{
      \begin{tabular}{l}
        Autonomous UAV\\
        landing on a car\\
        \emph{ECMR 2017}
      \end{tabular}
    }

    \keyarticle{right=of baca2017autonomous}{baca2019autonomous}{\tabcite{baca2019autonomous}}{
      \begin{tabular}{l}
        Autonomous UAV\\
        landing on a car\\
        \emph{JFR 2019}
      \end{tabular}
    }

    \article{below=of $(rospix_technology.south |- baca2019autonomous.south)$,shift={(0em, -1em)}}{baca2018rospix}{\tabcite{baca2018rospix}}{
      \begin{tabular}{l}
        Rospix: Timepix\\
        interface for Robot\\
        Operating System\\
        \emph{JINST 2018}
      \end{tabular}
    }

    \article{right=of baca2018rospix}{daniel2017xray}{\tabcite{daniel2017xray}}{
      \begin{tabular}{l}
        X-Ray telescope\\
        for a sounding\\
        rocket experiment\\
        \emph{SPIE O\&P 2017}
      \end{tabular}
    }

    \keyarticle{below=of $(mpc_tracker_technology.south |- loianno2018localization.south)$}{baca2018model}{\tabcite{baca2018model}}{
      \begin{tabular}{l}
        Model Predictive\\
        Control Tracker\\
        for UAVs\\
        \emph{IROS 2018}
      \end{tabular}
    }

    \technology{below=of $(baca2018model.south |- baca2017autonomous.south)$,shift={(0em, -1em)}}{mrs_uav_system_technology}{
      \begin{tabular}{l}
        MRS UAV System
      \end{tabular}
    }

    \article{below=of baca2017autonomous.south,shift={(0em, -1em)}}{faigl2017onsolution}{\tabcite{faigl2017onsolution}}{
      \begin{tabular}{l}
        On solution of\\
        Dubins touring\\
        problem\\
        \emph{ECMR 2017}
      \end{tabular}
    }

    \article{below=of $(mrs_uav_system_technology.south |- faigl2017onsolution.south)$}{petrlik2020robust}{\tabcite{petrlik2020robust}}{
      \begin{tabular}{l}
        Robust UAV system\\
        for constrained\\
        environment\\
        \emph{RA-L 2020}
      \end{tabular}
    }

    \article{below=of faigl2017onsolution,shift={(0em, 0em)}}{roucek2019darpa}{\tabcite{roucek2019darpa}}{
      \begin{tabular}{l}
        DARPA SubT\\
        mine exploration\\
        by Robots\\
        \emph{MESAS 2019}
      \end{tabular}
    }

    \subarticle{right=of roucek2019darpa,shift={(0.0em, 0.0em)}}{kratky2020autonomous2}{\tabcite{kratky2020autonomous2}}{
      \begin{tabular}{l}
        DARPA Subt\\
        Urban exploration\\
        \emph{JFR 2020}
      \end{tabular}
    }

    \article{below=of roucek2019darpa,shift={(0em, 0em)}}{saikin2020wildfire}{\tabcite{saikin2020wildfire}}{
      \begin{tabular}{l}
        Wildlife firefighting\\
        with UAVs\\
        \emph{RA-L 2020}
      \end{tabular}
    }

    \subarticle{right=of saikin2020wildfire,shift={(0em, 0em)}}{silano2020power}{\tabcite{silano2020power}}{
      \begin{tabular}{l}
        Powerline inspection\\
        with UAVs using\\
        STL\\
        \emph{RA-L 2020}
      \end{tabular}
    }

    \article{right=of faigl2017onsolution,shift={(0em, 0em)}}{giernacki2019realtime}{\tabcite{giernacki2019realtime}}{
      \begin{tabular}{l}
        Real-time UAV\\
        controller tuning\\
        \emph{Sensors 2019}
      \end{tabular}
    }

    \article{below=of saikin2020wildfire,shift={(0em, 0em)}}{petracek2020bioinspired}{\tabcite{petracek2020bioinspired}}{
      \begin{tabular}{l}
        Bio-inspired\\
        compact UAV\\
        swarms\\
        \emph{B\&B 2020}\\
      \end{tabular}
    }

    \subkeyarticle{below=of $(petrlik2020robust.south |- petracek2020bioinspired.south)$,shift={(0em, 0em)}}{baca2020mrs}{\tabcite{baca2020mrs}}{
      \begin{tabular}{l}
        MRS UAV system\\
        for research\\
        evaluation\\
        \emph{JINT 2020}
      \end{tabular}
    }

    \article{right=of petracek2020bioinspired,shift={(0em, 0em)}}{saska2020formation}{\tabcite{saska2020formation}}{
      \begin{tabular}{l}
        Formations of UAVs\\
        straitened\\
        environments\\
        \emph{AuRo, 2020}\\
      \end{tabular}
    }

    \subarticle{below=of $(daniel2017xray.south |- saska2020formation.south)$}{urban2020rex}{\tabcite{urban2020rex}}{
      \begin{tabular}{l}
        X-Ray telescope\\
        on NASA's sounding\\
        rocket experiment\\
        \emph{AA 2020}
      \end{tabular}
    }

    \subarticle{below=of petracek2020bioinspired,shift={(0em, 0em)}}{ahmad2020autonomous}{\tabcite{ahmad2020autonomous}}{
      \begin{tabular}{l}
        Autonomous UAV\\
        swarms without\\
        GNSS and comm.\\
        \emph{ICRA, 2021}\\
      \end{tabular}
    }

    \subarticle{right=of ahmad2020autonomous,shift={(0em, 0em)}}{dmytruk2020safe}{\tabcite{dmytruk2020safe}}{
      \begin{tabular}{l}
        Tightly-constrained\\
        UAV swarms\\
        without GNSS\\
        \emph{ICRA, 2021}\\
      \end{tabular}
    }

    \keyarticle{below=of $(baca2018rospix.south |- faigl2017onsolution.south)$,shift={(-4.6em, 0.0em)}}{baca2019timepix}{\tabcite{baca2019timepix}}{
      \begin{tabular}{l}
        Radiation mapping\\
        with Timepix sensor\\
        onboard UAV\\
        \emph{IROS 2019}
      \end{tabular}
    }

    \keyarticle{below=of baca2019timepix,shift={(0.0em, 0em)}}{stibinger2020localization}{\tabcite{stibinger2020localization}}{
      \begin{tabular}{l}
        Radiation mapping\\
        with Timepix sensor\\
        by UAVs\\
        \emph{RA-L 2020}
      \end{tabular}
    }

    \subarticle{below=of $(stibinger2020localization.south |- saska2020formation.south)$}{baca2020gamma}{\tabcite{baca2020gamma}}{
      \begin{tabular}{l}
        $\gamma$-source localization\\
        by a UAV with\\
        a Compton camera\\
        \emph{ICRA 2020}
      \end{tabular}
    }

    \subarticle{below=of $(ahmad2020autonomous.south |- baca2020mrs.south)$,shift={(-4.0em, -1.0em)}}{walter2020extinguishing}{\tabcite{walter2020extinguishing}}{
      \begin{tabular}{l}
        Extinguishing of\\
        ground fires\\
        by UAVs\\
        \emph{ICRA, 2021}
      \end{tabular}
    }

    \subarticle{right=of walter2020extinguishing,shift={(0.0em, -0em)}}{stasinchuk2020multiuav}{\tabcite{stasinchuk2020multiuav}}{
      \begin{tabular}{l}
        Autonomous\\
        target elimination\\
        by UAVs\\
        \emph{ICRA, 2021}
      \end{tabular}
    }

    \subarticle{below=of baca2020mrs,shift={(0.0em, -1em)}}{smrcka2020admittance}{\tabcite{smrcka2020admittance}}{
      \begin{tabular}{l}
        Admittance UAV\\
        stabilization for\\
        building inspection\\
        \emph{RA-L 2020}
      \end{tabular}
    }

    \subarticle{right=of stasinchuk2020multiuav,shift={(0.0em, -0em)}}{spurny2020autonomous}{\tabcite{spurny2020autonomous}}{
      \begin{tabular}{l}
        Autonomous UAV\\
        firefighting inside\\
        a building\\
        \emph{IEEE Access, 2020}
      \end{tabular}
    }

    \subarticle{below=of walter2020extinguishing,shift={(0.0em, -0em)}}{vrba2020autonomous}{\tabcite{vrba2020autonomous}}{
      \begin{tabular}{l}
        Autonomous target\\
        capturing by a UAV\\
        \emph{SMCA 2020}
      \end{tabular}
    }

    \subarticle{right=of vrba2020autonomous,shift={(0.0em, -0em)}}{baca2020autonomous}{\tabcite{baca2020autonomous}}{
      \begin{tabular}{l}
        Autonomous wall\\
        building by a group\\
        of UAVs\\
        \emph{RAS 2020}
      \end{tabular}
    }

    %%}

    %%{ backgrounds

    \begin{pgfonlayer}{background}
      \path (spurny2019cooperative.east |- spurny2019cooperative.north)+(+0.2, 0.2) node (a) {};
      \path (baca2017autonomous.south -| baca2017autonomous.west)+(-0.2,-0.4) node (b) {};
      \path[fill=black!0, draw=black!50, dashed]
      (a) rectangle (b);
      \path ($(a |- b)$) -- node [midway, shift = {(0.0, 1.5em)}] {\begin{tabular}{c}
        \tiny MBZIRC 2017\\
      \end{tabular}} ($(b |- b)$);
    \end{pgfonlayer}

    \begin{pgfonlayer}{background}
      \path (spurny2020autonomous.east |- spurny2020autonomous.north)+(+0.2, 0.2) node (a) {};
      \path (vrba2020autonomous.south -| vrba2020autonomous.west)+(-0.2,-0.4) node (b) {};
      \path[fill=black!00, draw=black!50, dashed]
      (a) rectangle (b);
      \path ($(a |- b)$) -- node [midway, shift = {(0.0, 1.5em)}] {\begin{tabular}{c}
        \tiny MBZIRC 2020\\
      \end{tabular}} ($(b |- b)$);
    \end{pgfonlayer}

    \begin{pgfonlayer}{background}
      \path (saska2020formation.east |- saska2020formation.north)+(+0.2, 0.2) node (a) {};
      \path (ahmad2020autonomous.south -| ahmad2020autonomous.west)+(-0.2,-0.4) node (b) {};
      \path[fill=black!00, draw=black!50, dashed]
      (a) rectangle (b);
      \path ($(a |- b)$) -- node [midway, shift = {(0.0, 1.5em)}] {\begin{tabular}{c}
        \tiny UAV Formations and Swarms\\
      \end{tabular}} ($(b |- b)$);
    \end{pgfonlayer}

    %%}

  %%{ paths

    \path[-] (baca2016miniaturized) edge [->] (daniel2016terrestrial);

    \path[-] (baca2018timepix) edge [->] (rospix_technology);
    \path[-] (rospix_technology) edge [->] (baca2018rospix);
    \path[-] (baca2018timepix) edge [->] (urban2017vzlusat);
    \path[-] (urban2017vzlusat) edge [->] (daniel2019inorbit);

    \path[-] (loianno2018localization) edge [->] (spurny2019cooperative);
    \path[-] (baca2017autonomous) edge [->] (baca2019autonomous);

    \path[-] (daniel2017xray) edge [->] (urban2020rex);

    \draw[-] ($(baca2016miniaturized.west)$) -- ($(radiation_stream.north |- baca2016miniaturized.east) + (-4.5em, 0.0em)$) edge [->] ($(radiation_stream.north) + (-4.5em, 0.0em)$);

    \path[-] (saska2013adhoc) edge [->] (chudoba2014localization);
    \path[-] (chudoba2014localization) edge [->] (baca2016embedded);
    \path[-] (baca2016embedded) edge [->] (saska2017documentation);
    \draw[-] ($(baca2016embedded.south)$) -- ($(baca2016embedded.south) + (0, -0.5em)$) -- ($(chudoba2016exploration.north |-baca2016embedded.south) + (0, -0.5em)$) edge [->] (chudoba2016exploration.north);
    \draw[-] ($(baca2016embedded.south)$) -- ($(baca2016embedded.south) + (0, -0.5em)$) -- ($(uav_platform_stream.north |-baca2016embedded.south) + (-2.0em, -0.5em)$) edge [->] ($(uav_platform_stream.north) + (-2.0em, 0em)$);
    \draw[-] ($(baca2016embedded.south)$) -- ($(baca2016embedded.south) + (0, -0.5em)$) -- ($(saska2016formations.north |-baca2016embedded.south) + (0, -0.5em)$) edge [->] (saska2016formations.north);
    \draw[-] ($(baca2016embedded.south)$) -- ($(baca2016embedded.south) + (0, -0.5em)$) -- ($(spurny2016complex.north |-baca2016embedded.south) + (0, -0.5em)$) edge [->] (spurny2016complex.north);
    \draw[-] ($(baca2016embedded.south)$) -- ($(baca2016embedded.south) + (0, -0.5em)$) -- ($(saska2017system.north |-baca2016embedded.south) + (0, -0.5em)$) edge [->] (saska2017system.north);

    \draw[-] ($(chudoba2016exploration.south)$) -- ($(chudoba2016exploration.south) + (0, -0.5em)$) -- ($(uav_sensing_stream.north |-chudoba2016exploration.south) + (0, -0.5em)$) edge [->] (uav_sensing_stream.north);
    \draw[-] ($(saska2016formations.south)$) -- ($(saska2016formations.south) + (0, -0.5em)$) -- ($(uav_sensing_stream.north |-saska2016formations.south) + (0, -0.5em)$) edge [->] (uav_sensing_stream.north);
    \draw[-] ($(spurny2016complex.south)$) -- ($(spurny2016complex.south) + (0, -0.5em)$) -- ($(uav_sensing_stream.north |-spurny2016complex.south) + (0, -0.5em)$) edge [->] (uav_sensing_stream.north);
    \draw[-] ($(saska2017system.south)$) -- ($(saska2017system.south) + (0, -0.5em)$) -- ($(uav_sensing_stream.north |-saska2017system.south) + (0, -0.5em)$) edge [->] (uav_sensing_stream.north);

    \draw[<-] ($(mrs_uav_system_technology.east)$) -- ($(mrs_uav_system_technology.east) + (2.2em, 0.0em)$) -- ($(mrs_uav_system_technology.east) + (2.2em, 0.0em)$) edge [->] ($(mrs_uav_system_technology.east |- stasinchuk2020multiuav.north) + (2.2em, 0.6em)$);
    \draw[-] ($(mrs_uav_system_technology.east)$) -- ($(mrs_uav_system_technology.east) + (2.2em, 0.0em)$) -- ($(mrs_uav_system_technology.east) + (2.2em, 0.0em)$) -- ($(mrs_uav_system_technology.east |- saikin2020wildfire.west) + (2.2em, 0.0em)$) edge [->] ($(saikin2020wildfire.west) + (0.0em, 0.0em)$);
    \draw[-] ($(mrs_uav_system_technology.east)$) -- ($(mrs_uav_system_technology.east) + (2.2em, 0.0em)$) -- ($(mrs_uav_system_technology.east) + (2.2em, 0.0em)$) -- ($(mrs_uav_system_technology.east |- petracek2020bioinspired.west) + (2.2em, 0.0em)$) edge [->] ($(petracek2020bioinspired.west) + (-0.6em, 0.0em)$);
    \draw[-] ($(mrs_uav_system_technology.east)$) -- ($(mrs_uav_system_technology.east) + (2.2em, 0.0em)$) -- ($(mrs_uav_system_technology.east) + (2.2em, 0.0em)$) -- ($(mrs_uav_system_technology.east |- smrcka2020admittance.north) + (2.2em, 1.2em)$) -- ($(smrcka2020admittance.north) + (0.0em, 1.2em)$) edge [->] ($(smrcka2020admittance.north) + (0.0em, 0.0em)$);
    \draw[-] ($(mrs_uav_system_technology.east)$) -- ($(mrs_uav_system_technology.east) + (2.2em, 0.0em)$) -- ($(mrs_uav_system_technology.east |- giernacki2019realtime.south) + (2.2em, -0.5em)$) -- ($(giernacki2019realtime.south) + (0.0em, -0.5em)$) edge [->] ($(giernacki2019realtime.south) + (0.0em, 0.0em)$);
    \draw[-] ($(mrs_uav_system_technology.east)$) -- ($(mrs_uav_system_technology.east) + (2.2em, 0.0em)$) -- ($(mrs_uav_system_technology.east |- silano2020power.north) + (2.2em, 0.5em)$) -- ($(silano2020power.north) + (0.0em, 0.5em)$) edge [->] ($(silano2020power.north) + (0.0em, 0.0em)$);

    \draw[-] ($(urban2017vzlusat.east)$) -- ($(urban2017vzlusat.east) + (1.0em, 0.0em)$) -- ($(urban2017vzlusat.east) + (1.0em, 0.0em)$) -- ($(urban2017vzlusat.east |- daniel2017xray.east) + (1.0em, 0.0em)$) edge [->] ($(daniel2017xray.east) + (0.0em, 0.0em)$);

    \draw[-] ($(rospix_technology.south)$) -- ($(rospix_technology.south) + (0, -0.5em)$) -- ($(baca2019timepix.north |-rospix_technology.south) + (0, -0.5em)$) edge [->] (baca2019timepix.north);

    \path[-] (baca2019timepix) edge [->] (stibinger2020localization);
    \path[-] (mrs_uav_system_technology.east) edge [<->] ($(faigl2017onsolution.west |- mrs_uav_system_technology.east)$);
    \path[-] (roucek2019darpa) edge [->] (kratky2020autonomous2);
    \path[-] (petrlik2020robust.east) edge [<->] (roucek2019darpa.west);

    \draw[-] ($(rospix_technology.east) + (0.0em, 0.0em)$) -- ($(rospix_technology.east) + (0.5em, 0.0em)$) -- ($(rospix_technology.east |- baca2020gamma.west) + (0.5em, 0.0em)$) edge [->] (baca2020gamma.east);
    \draw[-] ($(rospix_technology.east) + (0.0em, 0.0em)$) -- ($(rospix_technology.east) + (0.5em, 0.0em)$) -- ($(rospix_technology.east |- urban2020rex.west) + (0.5em, 0.0em)$) edge [->] (urban2020rex.west);

    \path[-] (mpc_tracker_technology) edge [->] (baca2018model);
    \path[-] (baca2018model) edge [->] (mrs_uav_system_technology);
    \path[-] (mrs_uav_system_technology) edge [->] (petrlik2020robust);
    \path[-] (petrlik2020robust) edge [->] (baca2020mrs);
    \path[-] (mpc_tracker_technology) edge [<->] ($(loianno2018localization.west |- mpc_tracker_technology.east) + (-0.6em, 0)$);

  %%}

  \end{tikzpicture}

  \end{adjustbox}

  \caption{
    Diagram of research performed by the thesis author from 2013 to 2020 in the fields of UAV control, remote sensing and its applications, and the field of ionizing radiation imaging, dosimetry, mapping and localization.
    The article numbering reflects the works in the reference section of this thesis.
  }

  \label{fig:research_graph}

\end{figure}

%%}

%%{ UAV control system

\section{Multirotor UAV control system}

Commercial \ac{UAV} systems are often closed-source and provide features tailored for photographers, video makers and hobby pilots.
Autonomous operation of commercial \acp{UAV} is typically limited to a single vehicle flying outdoors under a \ac{GNSS} localization while following a set of waypoints.
Therefore, commercial platforms are rarely used for research and if so, then in field where the \ac{UAV} is only considered as a \emph{sensor carrier}, without added onboard autonomy.

Research-focused \acp{UAV} are most commonly equipped with a low-level embedded flight controller.
Available flight controllers \cite{ebeid2018survey} range from feature-packed open-source systems, such as Pixhawk, to proprietary commercial units manufactured by DJI.
Table~\ref{tab:embedded_flight_controllers} shows a comparison of often used solutions.
Pixhawk is often used in research projects (including our project), typically running either of the two open-source firmwares: PX4 \cite{meier2015px4} and ArduPilot\footnote{\url{http://ardupilot.org}}.
Although all of these flight stacks provide sophisticated features up to waypoint tracking and mission execution, the features are rarely used within real-world applications.
Instead, researchers use other onboard computers to execute a custom localization system, state estimators, and flight controllers, and only low-level control commands are provided for the embedded flight controller.
% Thus the choice of the embedded flight controller is not a crucial one as an adaption of the high-level system to a different underlying platform is often just a matter of implementation detail.

Several comparable \ac{UAV} systems have been published and released.
Table~\ref{tab:uav_systems} compares existing solution with the system proposed in this publication.

The RotorS \cite{furrer2016rotors} simulator is an initial release for the Aeroworks EU project\footnote{Aeroworks EU project, \url{http://www.aeroworks2020.eu}.}.
It provides Gazebo-based simulation of the now discontinued \emph{Ascending Technologies} \ac{UAV} system.
The control pipeline features are basic, with little potential for transfer to real-world conditions.
The system does not appear to be kept up-to-date, which gradually diminishes its usability and applicability.
Moreover, the latest supported version of ROS is \emph{ROS Kinetic}, which potentially provides lower compatibility with newer hardware and software.

OpenUAV\footnote{OpenUAV, \url{http://github.com/Open-UAV}} \cite{schmittle2018openuav} is a \ac{UAV} swarm simulation testbed.
The system does not appear to allow transfer to a real-world setting, and is designed only to support prototyping of basic research in swarming.
The \acp{UAV} are assumed to be controlled and localized solely using an embedded flight controller with PX4 firmware.
This is comparable hardly with the numerous sensors and localization systems that our system allows to simulate and to be used in a real-world scenario.

ReCOPTER\footnote{ReCOPTER, \url{http://github.com/thedinuka/ReCOPTER}} \cite{abeywardena2015design} proposes an open-source multirotor system for research.
The available materials were released as supporting material for the published paper.
However, no software was attached, and the materials have not been updated since.
Similarly, a framework for drone control using the Vicon localization system named MAVwork\footnote{MAVwork, \url{http://github.com/uavster/mavwork}} \cite{mellado2013mavwork} was published in 2011, but has not been updated since.
Although sources were made available, they offered only basic features that would be difficult to transfer into a real-world scenario.

The XTDrone\footnote{XTDrone, \url{http://github.com/robin-shaun/XTDrone}} \cite{xiao2020xtdrone} simulation testbed offers many complex functionalities that are comparable with our proposed system, including simulation of onboard sensors and complex localization systems.
However, the control pipeline relies entirely on the PX4 embedded control software.
This significantly limits any transfer to a custom hardware platform, or even the ability to simulate realistic conditions using onboard localization systems.
Thus, the use of XTDrone outside laboratory conditions is mostly limited to \ac{GPS}-localized flight in a non-cluttered outdoor environment.

The full-stack Aerostack system\footnote{Aerostack, \url{http://github.com/Vision4UAV/Aerostack}} \cite{sanchez2016aerostack, sanchez2016reliable} was designed for deployment of multirotor \acp{UAV}.
The system is continuously being updated, and it offers an option to transfer to a real-world platform.
According to the preprint\cite{suarez2020skyeye} where the authors used Aerostack during the MBZIRC 2020 competition, the system's real-world deployment is possible.
However, with the used DJI-based flight controller, the control command supplied to the underlying embedded control layer are limited to desired orientation and thrust.
This level of control limits the potential precision and control authority comparing to our system.
Furthermore, the system lacks the feature of switching between multiple frames of reference, which is one of our system's contributions.
As it happens, the team of authors of Aerostack did not compete in the wall-building challenge of MBZIRC 2020, in which we found the feature to be crucial to precisely collect bricks by a group of \acp{UAV}.

Besides the Aerostack system, no other existing platform provides a full-stack system for a multirotor \ac{UAV} that is actively being supported and updated.
Many publications provide accompanying software sources that are released without being further updated.
By contrast, we have decided to publish and release our working system with all its components to allow members of the research community, research teams, and students to engage in \ac{UAV} research as effortlessly as possible.
We aim to provide a thoroughly-documented open-source system to allow researchers and students to shorten their initial learning curve and to focus on their research instead of developing yet another control pipeline.
% In our case, the future continuity of our system is supported for use in the next 5+ years through our numerous activities in projects supported by European grants\footnote{\url{https://aerial-core.eu}, \url{http://rci.cvut.cz}} and by national grants\footnote{\url{http://mrs.felk.cvut.cz}}.

The proposed platform is provided with two control designs --- extended \emph{SE(3) geometric tracking} \cite{lee2010geometric} for agile and aggressive flight, and the novel \emph{MPC controller} for stable flight using a potentially unreliable state estimate.
However, we highlight the modularity of our platform, which can easily be extended with new control approaches as needed.
The survey of UAV controllers provides a rich list of potentially useful control techniques \cite{nascimento2019position}.
For example, a novel adaptive backstepping controller \cite{zhang2019robust, labbadi2019robust} may provide better performance during aggressive maneuvers, thanks to the included rotor drag compensation.
The proposed extension to \emph{geometric tracking on SE(3)} \cite{lee2010geometric} can be further improved with remarks from \cite{lee2013nonlinear} to provide robust control to bounded uncertainties.
Furthermore, nonlinear \ac{MPC} controllers are becoming popular \cite{nascimento2019nmpc, pereira2019nonlinear, kamel2017robust}, thanks to their inherent ability to deal with obstacle avoidance.
However, when dealing not just with theoretical work but also with the deployment of \acp{UAV} in real-world conditions, we favor practicality over complexity.
We therefore propose the use of relatively simple controllers \cite{baca2020mrs}, with well tractable inner mechanisms.

%%{ Tab: embedded controllers

\begin{table}
  \centering
  \def\arraystretch{0.8}% 1 is the default, change whatever you need
  \setlength\tabcolsep{1.5pt} % default value: 6pt
  \begin{tabular}{c||c|c|c|c|c}
    \hline\noalign{\smallskip}
    \scriptsize platform         & \begin{tabular}{@{}c@{}} \scriptsize open \\ [-0.3em] \scriptsize source\end{tabular} & \scriptsize modular                     & \scriptsize SITL/HITL  & \begin{tabular}{@{}c@{}} \scriptsize outside \\ [-0.3em] \scriptsize lab\end{tabular} & \begin{tabular}{@{}c@{}} \scriptsize rate \\ [-0.3em] \scriptsize input\end{tabular} \\
      \noalign{\smallskip}\hline\hline\noalign{\smallskip}
    \scriptsize \textbf{Pixhawk} & \scriptsize \textbf{SW \& HW}              & \scriptsize \textbf{+}                   & \scriptsize \textbf{+}                  & \scriptsize \textbf{+} & \scriptsize \textbf{+}                       \\
    \scriptsize DJI              & \scriptsize -                              & \scriptsize -                            & \scriptsize - \scriptsize (proprietary) & \scriptsize +          & \scriptsize \textbf{-}                       \\
    \scriptsize Ardupilot        & \scriptsize SW                             & \scriptsize +                            & \scriptsize +                           & \scriptsize +          & \scriptsize \textbf{-}                       \\
    \scriptsize Parrot           & \scriptsize SW                             & \scriptsize -                            & \scriptsize +                           & \scriptsize -          & \scriptsize \textbf{-}                       \\
    \noalign{\smallskip}\hline
  \end{tabular}
  \caption{Comparison of commonly-used embedded flight controllers and low-level control systems. The Pixhawk flight controller was chosen due to several factors: both the hardware and software is open-source, the controller is modular enough to be used on a variety of custom multirotor platforms, Pixhawk supports both hardware- and software-in-the-loop simulation, can be used outside of laboratory conditions, and supports attitude rate input.\label{tab:embedded_flight_controllers}}
\end{table}

%%}

%%{ Tab: UAV systems

\begin{table*}
  \centering
  \def\arraystretch{0.8}%  1 is the default, change whatever you need
  \setlength\tabcolsep{4.0pt} % default value: 6pt
  \begin{tabular}{c|c|c|c|c|c|c|c}
\noalign{\smallskip}\hline
    \small platform                       & \small modular    & \small simulation & \begin{tabular}{@{}c@{}} \small outside \\ [-0.1em] \small lab.\end{tabular} & \begin{tabular}{@{}c@{}} \small multi-frame \\ [-0.1em] \small localization\end{tabular}  & \begin{tabular}{@{}c@{}} \small rate \\ [-0.1em] \small output\end{tabular} & \begin{tabular}{@{}c@{}} \small last \\ [-0.1em] \small updated\end{tabular} & \small reference \\
      \noalign{\smallskip}\hline\hline\noalign{\smallskip}
    \small \small \textbf{MRS UAV system} & \small \textbf{+}                   & \small \textbf{+} & \small \textbf{+} & \small \textbf{+}                        & \small \textbf{+}                       & \small \textbf{2020}                         & \cite{baca2020mrs}                                       \\
    \small Aerostack                      & \small +                            & \small +          & \small +          & \small -                                 & \small -                                & \small 2020                                  & \small \cite{sanchez2016aerostack}        \\
    \small XTDrone                        & \small +                            & \small +          & \small -          & \small -                                 & \small -                                & \small 2020                                  & \small \cite{xiao2020xtdrone}             \\
    \small RotorS                         & \small +                            & \small +          & \small -          & \small -                                 & \small +                                & \small 2020                                  & \small \cite{furrer2016rotors}            \\
    \small OpenUAV                        & \small -                            & \small +          & \small -          & \small -                                 & \small -                                & \small 2020                                  & \small \cite{schmittle2018openuav}        \\
    \small ReCOPTER                       & \small -                            & \small -          & \small -          & \small -                                 & \small -                                & \small 2015                                  & \small \cite{abeywardena2015design}       \\
    \small MAVwork                        & \small +                            & \small -          & \small -          & \small -                                 & \small -                                & \small 2013                                  & \small \cite{mellado2013mavwork}          \\
\noalign{\smallskip}\hline
  \end{tabular}
  \caption{Comparison of high-level open-source \ac{UAV} systems. The proposed system is extensible and modular, comes with an extensive simulation environment, is designed to be used outside of laboratory conditions, provides the novel multi-frame localization estimator, and supplies the attitude rate command to the underlying embedded flight controller.\label{tab:uav_systems}}
\end{table*}

%%}

The proposed system goes beyond existing systems with
\begin{itemize}
  \item a novel bank-of-filters estimator design that overcomes challenges with diverse sensory equipment,
  \item a heading-oriented control design, devoid of ambiguous use of Euler/Tait-Bryan angles,
  \item a body/world disturbance estimation approach that does not rely on a specific state estimator design,
  \item a reliable MPC-based controller with the benefits of the nonlinear SO(3) force feedback,
  \item a system that can be employed with a variety of onboard localization systems and sensors,
  \item an ability to supply references in coordinate frames, which differ from the feedback loop reference frame.
\end{itemize}

The system is not only innovative, but also provides practical contributions to the community.
The open-source implementation\footnote{\url{http://github.com/ctu-mrs/mrs_uav_system}} of the proposed platform has been tested extensively in real-world settings and in conditions of outdoor fields, in a forest, indoors, in a factory, in mines, caves and tunnels, during object manipulation, during fast and aggressive flights, and in autonomous landing on a moving platform.
The system includes a simulation environment based on the Gazebo 3D simulator with realistic sensors and models that can be run in real time.
% The released platform is fully compatible with multiple releases of ROS (Melodic, Noetic), and is being actively used and maintained.
The system is scalable for multiple \acp{UAV} and is well suited for research in swarming.

%%}

%%{ remote sensing

\section{Remote sensing and data collection by UAVs}

Remote sensing by \acp{UAV} has been common since the first aerial platforms become available \cite{colomina2014unmanned, pajares2015overview}.
Foliage and vegetation monitoring were among the first applications \cite{barrientos2011aerial}.
In earlier attempts, thermal cameras provided information that was later used to make educating actions during farming in the scanned field \cite{berni2009thermal} .
Also, water status monitoring within a vineyard \cite{baluja2012assessment} can be conducted by multi-spectral cameras onboard a \ac{UAV}.
Nowadays, precise agriculture is a promising agricultural technique that utilizes multispectral cameras to decide upon adjustments within the agriculture processes \cite{fu2020wheat, zha2020improving, jang2020cost}.
However, it is often conducted by a single \ac{UAV} dedicated to monitoring the field from a birds-eye perspective or to deliver chemicals to precise locations within the field.

Distributing the remote sensing task provides measurements in different places at once.
Multiple UAVs equipped with gimballed cameras can track a ground target more reliable than a single \ac{UAV} \cite{sun2014distributed}.
The redundancy increases the robustness of the system.
Missing information due to, e.g., image dropouts or target occlusions, can by substituted in real time by communication with neighboring \acp{UAV} \cite{baek2020optimal, farmani2015tracking}.
Moreover, distributed state estimation can be applied in real time \cite{capitan2009delayed, merino2007multi, wan2000unscented}.
When the states of an aerial target are estimated in real time, a group of \acp{UAV} can pursue the target and intercept its path \cite{zhu2017distributed}.

A likely application of distributed sensing is during rescue operations and environment monitoring of natural disasters \cite{manfreda2018use}.
Distributed monitoring of wildfires can provide crucial real-time information to firefighters \cite{yuan2015survey, pham2020distributed}.
Similarly, monitoring the current state of floods can aid during rescue operations \cite{karamuz2020use, perks2016advances}.

Deploying multiple \acp{UAV} simultaneously poses challenges on various levels of the onboard autonomy architecture: automatic control, localization, planning, communication, and collision avoidance.
Multi-\ac{UAV} path planning and coordination algorithms provide the aircraft with feasible plans in order to fulfill the given task.
The coordinate principles are often split into two categories: \emph{formation} and \emph{swarm} control.
\ac{UAV} formation control focuses on explicit coordination given either a centralized localization of all the agents or a mutual communication between the agents \cite{kamel2017model, kuriki2014avoidance, rezaee2013motion, wang2015efficient}.
On the other hand, \ac{UAV} swarms is a bio-inspired decentralized technique which attempts to overcome the global localization or communication by applying a set of behavioral rules \cite{mcguire2019minimal, burkle2011towards, saulnier2017flocking}.

We aim at the specific sub-field of remote sensing related to autonomous experimental deployment of aerial robots.
In particular, two scenarios were studied in context of the \ac{MBZIRC} 2017 challenge \cite{dias2019journal}.
The first scenario focuses on autonomous localization, transportation and delivery of objects by a team of \acp{UAV}.
Furthermore, we investigate the task of autonomously landing a \ac{UAV} on a moving vehicle, potentially ending an automated sensing mission.
Prior to our achievements \cite{loianno2018localization, spurny2019cooperative}, distributed object gathering \acp{UAV} has only been attempted in laboratory conditions, mostly in relation to automatic assembly of structures \cite{mirjan2016building, alejo2014collisionfree, augugliaro2014flight, lindsey2013distributed}.
The state-of-the art often focused on a subset of tasks needed for full autonomy, e.g., the autonomous grasping \cite{gawel2017aerial, thomas2014toward} and motion planning \cite{alejo2014collisionfree}.
Other sub-tasks such as onboard detections of the objects, \ac{UAV} localization or the full mission autonomy are often omitted.
The \ac{MBZIRC} 2017 challenged forced the research groups to leave ideal lab conditions, show a fully-integrated robotics system, and to provide repeatable results.
Only handful of the 147 research groups who applied \cite{beul2019team, bahnemann2019eth, lee2019mission} were able to complete the task of automatically grasping an object and delivering it to a desired location, with us winning the challenge \cite{spurny2019cooperative}.
The second challenge of autonomous landing on a moving car was also successfully tackled by a handful of groups \todo{}.
\todo{}

Methods of artificial intelligence will be involved to conduct not only pre-planned missions but also autonomous missions in which actions of the fleet are influenced by currently perceived information.
Control of a group of multiple aircraft introduces new challenges.
If visibility between the UAVs is used to estimate their relative positions or is otherwise relevant to the task, the formation as a whole needs to be treated as a dynamical system.
Authors of \cite{stipanovic2004decentralized} present a solution to a scenario, where aircraft require clear line of sight on the aircraft in front of them in the formation.
Decentralized control using sliding mode controllers was successfully tested in simulation.
Real world experiments of a system suitable for patrolling and surveillance have been conducted in \cite{ryan2007decentralized}.
Four \acp{UAV} with completely decentralized and distributed sensing and planning communicated using an ad-hoc wireless network.
A novel platform for designing and experimental verification of multi-\ac{UAV} systems was presented in \cite{sanchez2016reliable}.
Authors tackle various subproblems including control, planning, localization, communication and human machine interface.
Both simulation and experimental results present an autonomous flight of four \acp{UAV} in a small area with obstacles.
A solution with more attention to distributed measurement and estimation is presented in \cite{maza2011distributed}.
In addition to \cite{sanchez2016reliable}, distributed task allocation and task synchronization are studied and verified using real-world scenarios of forest fire monitoring, fireman tracking, surveillance, and node deployment.
% Distributed delayed-state information filter \cite{, merino2007multi} is used to estimate the targets in real-time.
% Distributed information filter was also used in \cite{} for distributed estimation of the visually detected target.
% Multiple \acp{UAV} cooperatively tracked the detected targets and estimated states of other potential targets in a grid.
% In the real world, communication delays often cause issues.
% Not only in estimation but also in autonomous control, communication delays have to be taken into account.
% Coupled sliding mode controller was presented in \cite{rezaee2013motion} to control a formation of \acp{UAV} using a virtual leader approach.
% The whole system is decentralized and was evaluated in simulations.

%%}

%%{ Radiation

\section{Measuring ionizing radiation, mapping and localization}

%%{ extraterrestria

\subsection{Extraterrestrial radiation dosimetry and mapping}

Ionizing radiation is commonly measured for two reasons in the constrained environment of space applications.
The first one is to asses the radiation dose deposited to crew members onboard and the onboard space electronics.
Traditional dosimeters on NASA's \ac{ISS} are being lately replaced with \emph{smart} pixel detectors \cite{turecek2011small, stoffle2015timepix, pinsky2019timepix}, namely, the Timepix sensor.
The Timepix sensor \cite{llopart2007timepix, poikela2014timepix3} is an innovative design of an \ac{ASIC} \ac{CMOS} chip that can be bonded to a variety of semi-conductor detection materials (Silicon, CdTe).
Despite it being originally developed for medical imaging \cite{ballabriga2018asic} and laboratory measurements, the Timepix has found applications even outside laboratory environments.
Timepix pixel detectors are unique for their capability to measure traces of incoming ionizing particles with the detector.
With those traces, machine learning algorithms can be used to deduce the particle type, the energy of each particle \cite{granja2018resolving, baca2019timepix}.
Long-term and large-scale capability of radiation dosimetry with Timepix detectors has been also tested in space outside of the \ac{ISS}.
Several satellites included the Timepix as their payloads: ESA's Proba-V \cite{granja2014directional, granja2016satram}, British TechDemoSat-2 \cite{furnell2018first}, and the Japanese RISESat \cite{filgas2019risepix}.

The second use of radiation detectors in space is dedicated to capturing X-Ray/Gamma ray photons through a focusing device of a telescope.
X-Ray observatories can not be deployed on the Earth's surface due to the presence of the atmosphere.
The state-of-the-art observatories such as Chandra \cite{weisskopf2000chandra}, Swift \cite{gehrels2004swift}, and Fermi \cite{atwood2009large} typically use scintillating detectors to measure the incoming light, or \ac{CCD} detectors for low-energy X-rays.
\ac{CMOS} detectors are rarely used in space applications, mainly due to undesired noise characteristics.
However, Timepix detectors poses a special mechanism to produce images completely without noise, which allows to capture even single photon event.

%%{ Fig: vzlusat example

\begin{figure}[!h]
  \centering
  \subfloat[VZLUSAT-1] {
    \includegraphics[width=0.48\textwidth,trim={1cm 2.5cm 0cm 2.5cm},clip]{./fig/plots/vzlusat_map.eps}
  }
  \subfloat[ISS, courtesy of \cite{stoffle2015timepix}] {
    \includegraphics[width=0.48\textwidth]{./fig/plots/iss_map_2.png}
  }
  \caption{Ionizing radiation dose measured by the Timepix sensor onboard (a) VZLUSAT-1 ($\approx 500$\,km Sun-synchronous orbit), and (b) the \acl{ISS} ($\approx 350$\,km orbit).}
  \label{fig:intro_radiation_maps}
\end{figure}

%%}

We add to the list an embedded and lite-weight design of a Timepix payload for the first Czech CubeSat satellite, the VZLUSAT-1 \cite{urban2017vzlusat, daniel2019inorbit}.
The payload was dedicated to radiation mapping of Earth's \ac{LEO} \cite{baca2018timepix} and capturing images using the onboard X-Ray telescope \cite{baca2016miniaturized}.
The satellite has been operational for more than 3 years.
New radiation data are being processed on regular basis while being added to an open-source dataset\footnote{\url{https://github.com/klaxalk/vzlusat-timepix-data}} potentially useful for the community of future CubeSat designers.
Further research let to a design of a one-time sub-orbital rocket experiment \cite{daniel2017xray, urban2020rex}.
A dedicated payload with two Timepix sensors were designed and developed based upon the \acl{ROS} \cite{baca2018rospix}.
Embedded hardware with \ac{ROS} automatically managed recording of the measured data in real time.
This crossover of technologies later allowed to continue on the development of \ac{ROS}-based Timepix technologies for the use onboard \aclp{UAV}.

%%}

%%{ terrestrial

\subsection{Localization and mapping of ionizing radiation sources by \acp{UAV}}

In the field of radiation sensing, unmanned robotic vehicles offer several advantages over conventional handheld detectors or piloted aircraft.
These advantages can be exploited in a wide variety of applications.

Following the 2011 disaster at the \ac{FDNPP}, considerable amounts of radioactive material have been released into the plant area.
Several \acp{UGV} have been deployed directly inside the damaged reactor buildings of \ac{FDNPP} under remote control.
Various radiation detection methods have been tested inside the power plant, including a coded aperture scintillator \cite{ohno2011robotic}, a semiconductor digital dosimeter \cite{nagatani2013emergency}, a Compton event camera composed of two scintillators \cite{sato2019radiation}, and a time-of-flight gamma camera \cite{kinoshita2014development}.
Ground-based robots offer higher payload capacity and the ability to carry heavier sensory equipment than most aerial vehicles.
On the other hand, these robots tend to be relatively bulky, struggle to navigate the cluttered corridors and staircases inside the damaged buildings, and generally move slower than a multirotor aircraft.

\acp{UAV} have been utilized to map the spread of the radioactive material outside of the power plant.
These range from large aircraft weighing more than \SI{90}{\kilogram} equipped with heavy scintillation detectors \cite{sanada2015aerial, towler2012radiation, jiang2016prototype}
to compact multi-rotors suitable for flying along a pre-defined trajectory close to the ground \cite{macfarlane2014lightweight, christie2017radiation, martin20163d}.
Outside of Japan, several projects have employed \acp{UAV} for radiation intensity mapping around uranium ore mines \cite{salek2018mapping, keatley2018source, martin2015use}.

In \cite{han2013lowcost}, multiple fixed-wing \acp{UAV} equipped with miniature scintillators are used for contour analysis of an irradiated area.
Trajectory planning and data processing are performed offline, contrary to our approach, which estimates the position of the source in real-time during the flight.
In \cite{newaz2016uav}, the contour analysis is tackled using a single multirotor \ac{UAV}. The contour analysis uses a Gaussian mixture model to estimate multiple radiation sources' positions with overlapping intensity fields.
The projects mentioned above utilize unmanned vehicles to deliver a radiation sensor into a hazardous environment.
However, the approaches do not respond to measured data in real-time and thus do not exploit the mobility of \acp{UAV} to improve the measurement.

Active path-planning driven by the onboard measurements has been shown in \cite{towler2012radiation} for an outdoor environment and in \cite{mascarich2018radiation} for a GPS-denied indoor environment.
Both of these works rely on a scintillator sensor to estimate the radiation intensity in the \ac{UAV}'s current position.
As a result, the employed aerial platforms have to be large with a payload capacity exceeding \SI{2}{\kilogram}.
As in \cite{towler2012radiation}, the aerial platform is a \SI{90}{\kilogram} unmanned aerial helicopter, which significantly limits its deployment conditions due to personal safety and considerable minimum distance to obstacles in the environment.

The lack of lightweight radiation detectors with immediate readout capability severely limits the application potential of aerial dosimetry.
However, the Timepix pixel detectors are ideal for the use onboard micro \acp{UAV} thanks to their low weight, small size and the absence of any active cooling mechanism.
We propose a \ac{UAV} system for outdoor and indoor environments while utilizing the know-how obtained with the work on the embedded space applications.
The \ac{ROS} interface for Timepix \cite{baca2018rospix}, originally developed for a suborbital rocket experiment \cite{urban2020rex}, has allowed the technology transfer to the robotics field.
However, a new robotic methodology for motion planning and exploration needed to be developed to accommodate and utilize the proposed measurement system's specifics \cite{stibinger2020localization}.
Moreover, the Compton camera mechanism \cite{turecek2018compton}, is being utilized to provide the smallest real-time single-sensor Compton camera ever used on a \acp{UAV} \cite{baca2020gamma}.

%%{ Fig: example

\begin{figure}[!h]
  \centering
  \subfloat[Visualization of the localization] {
    \includegraphics[width=0.415\textwidth]{./fig/plots/rviz_3d.png}
  }
  \subfloat[\ac{UAV} equipped with the Timepix3 sensor] {
    \includegraphics[width=0.415\textwidth]{./fig/photos/radron_uav.jpg}
  }
  \caption{Showcase from a radiation localization experiment with the MiniPIX Timepix3 Compton camera onboard a \ac{UAV}. The visualization shows the ground truth position of the radiation source (yellow) and the estimated position (red).}
  \label{fig:intro_uav_example}
\end{figure}

%%}

%%}

%%}

%%}

%% | ------------------------ Chapter 3 ----------------------- |

%%{ Research-focused UAV Platform

\chapternoclear{Research-focused UAV Platform\label{chap:uav_platform}}

\includepaperwithimage{baca2018model}
\includepaperwithimage{baca2020mrs}

%%}

%% | ------------------------ Chapter 4 ----------------------- |

%%{ Advances in remote sensing by UAVs

\chapternoclear{Advances in remote sensing by UAVs\label{chap:sensing}}

\includepaperwithimage{loianno2018localization}
\includepaperwithimage{spurny2019cooperative}
\includepaperwithimage{baca2019autonomous}

%%}

%% | ------------------------ Chapter 5 ----------------------- |

%%{ Ionizing Radiation Detection and Localization by UAVs

\chapternoclear{Ionizing Radiation Detection and Localization by UAVs\label{chap:radiation}}
\chaptermark{Ionizing Radiation Sources Localization}

Ongoing research realized in the accepted TACR 2020-2022 project FW01010317:\\
``\textit{Lokalizace zdrojů ionizující radiace pomocí malých bezpilotních helikoptér s detektorem na principu Comptonovy kamery}''.

% INCLUDE THE FOLLOWING PDFs
\includepaperwithimage{baca2018timepix}
\includepaperwithimage{baca2019timepix}
\includepaperwithimage{stibinger2020localization}

%%}

%% | ------------------------ Chapter 6 ----------------------- |

%%{ Conclusion

\chapternoclear{Conclusion}

\todo{}

%%}

%% | ----------------------- References ----------------------- |

%%{ References

\appendix
\renewcommand\chaptername{Appendix}

\chapternoclear{References}

Citations of the author's work are listed first, followed by other references cited within this work.
The author's citations were extracted from the Web of Science~(WoS) catalog.
First- and second-order self-citations were removed.
Each citation is displayed with the percentage contribution of the author and the number of citations based on WoS, Scopus and Google Scholar~(GS).
The author has reached \textbf{h-index} 7 at the time of writing the thesis.
\todo{updated citations}

%%{ Core publications

\section{Thesis core publications}

\subsection*{Core articles in peer-reviewed journals}
\printbibliography[keyword={mine},keyword={phd_related},keyword={journal},keyword={core},notkeyword={submitted},heading=none,title={}]

\subsection*{Core conference proceedings}
\printbibliography[keyword={mine},keyword={phd_related},keyword={conference},keyword={core},notkeyword={submitted},heading=none,title={}]

\subsection*{Core articles --- submitted}
\printbibliography[keyword={mine},keyword={phd_related},keyword={submitted},keyword={core},heading=none,title={}]

%%}

%%{ Related publications

\section{Thesis-related author's publications}

\subsection*{Thesis-related articles in peer-reviewed journals}
\printbibliography[keyword={mine},keyword={phd_related},keyword={journal},notkeyword={core},notkeyword={submitted},heading=none,title={}]

% \subsection*{Thesis-related articles in peer-reviewed journals only with CiteScore~(CS)}
% \printbibliography[keyword={mine},keyword={phd_related},keyword={journal},keyword={cs},heading=none,title={}]

\subsection*{Thesis-related conference proceedings}
\printbibliography[keyword={mine},keyword={phd_related},keyword={conference},notkeyword={core},notkeyword={submitted},heading=none,title={}]

\subsection*{Thesis-related publications --- submitted}
\printbibliography[keyword={mine},keyword={phd_related},keyword={submitted},notkeyword={core},heading=none,title={}]

%%}

%%{ Partially-related publications

\section{Partially-related author's publications}

\subsection*{Partially-related articles in peer-reviewed journals}
\printbibliography[keyword={mine},keyword={phd_unrelated},keyword={journal},notkeyword={core},notkeyword={submitted},heading=none,title={}]

\subsection*{Partially-related conference proceedings}
\printbibliography[keyword={mine},keyword={phd_unrelated},keyword={conference},notkeyword={core},notkeyword={submitted},heading=none,title={}]

\subsection*{Partially-related publications --- submitted}
\printbibliography[keyword={mine},keyword={phd_unrelated},keyword={submitted},heading=none,title={}]

%%}

%%{ unrelated publications

\section{Unrelated author's publications}

\printbibliography[keyword={mine},notkeyword={phd_unrelated},notkeyword={phd_related},heading=none,title={}]

%%}

%%{ Cited references

\section{Cited references}
\printbibliography[notkeyword=mine,heading=none,title={}]

%%}

%%}

%% | ------------------------ Apendices ----------------------- |

%%{ Apendices

\appendix
\renewcommand\chaptername{Citations of author's publications}

\chapternoclear{Citations of author's publications}

Citations of thesis author's publications were extracted from the Web of Science.
First- and second-order self-citations were excluded.

\DeclareCiteCommand{\fullcite}
{\usebibmacro{prenote}}
{\clearfield{addendum}%
  \usedriver
  {\defcounter{minnames}{6}%
  \defcounter{maxnames}{6}}
{\thefield{entrytype}}}
{\multicitedelim}
{\usebibmacro{postnote}}

\noindent
\fullcite{saska2017system}
\begin{refsection}[citations/no_autocit/saska2017system.bib]
  \nocite{*}
  \printbibliography[heading=none,title={},env=favoritebib]
\end{refsection}

% \noindent
% \fullcite{faigl2017onsolution}
% \begin{refsection}[citations/no_autocit/faigl2017onsolution.bib]
%   \nocite{*}
%   \printbibliography[heading=none,title={},env=favoritebib]
% \end{refsection}

% \noindent
% \fullcite{saska2016formations}
% \begin{refsection}[citations/no_autocit/saska2016formations.bib]
%   \nocite{*}
%   \printbibliography[heading=none,title={},env=favoritebib]
% \end{refsection}

\noindent
\fullcite{baca2016miniaturized}
\begin{refsection}[citations/no_autocit/baca2016miniaturized.bib]
  \nocite{*}
  \printbibliography[heading=none,title={},env=favoritebib]
\end{refsection}

\noindent
\fullcite{loianno2018localization}
\begin{refsection}[citations/no_autocit/loianno2018localization.bib]
  \nocite{*}
  \printbibliography[heading=none,title={},env=favoritebib]
\end{refsection}

\noindent
\fullcite{spurny2019cooperative}
\begin{refsection}[citations/no_autocit/spurny2019cooperative.bib]
  \nocite{*}
  \printbibliography[heading=none,title={},env=favoritebib]
\end{refsection}

\noindent
\fullcite{daniel2016terrestrial}
\begin{refsection}[citations/no_autocit/daniel2016terrestrial.bib]
  \nocite{*}
  \printbibliography[heading=none,title={},env=favoritebib]
\end{refsection}

\noindent
\fullcite{baca2017autonomous}
\begin{refsection}[citations/no_autocit/baca2017autonomous.bib]
  \nocite{*}
  \printbibliography[heading=none,title={},env=favoritebib]
\end{refsection}

\noindent
\fullcite{baca2018model}
\begin{refsection}[citations/no_autocit/baca2018model.bib]
  \nocite{*}
  \printbibliography[heading=none,title={},env=favoritebib]
\end{refsection}

\noindent
\fullcite{daniel2017xray}
\begin{refsection}[citations/no_autocit/daniel2017xray.bib]
  \nocite{*}
  \printbibliography[heading=none,title={},env=favoritebib]
\end{refsection}

\noindent
\fullcite{urban2017vzlusat}
\begin{refsection}[citations/no_autocit/urban2017vzlusat.bib]
  \nocite{*}
  \printbibliography[heading=none,title={},env=favoritebib]
\end{refsection}

\noindent
\fullcite{baca2016embedded}
\begin{refsection}[citations/no_autocit/baca2016embedded.bib]
  \nocite{*}
  \printbibliography[heading=none,title={},env=favoritebib]
\end{refsection}

\noindent
\fullcite{baca2019autonomous}
\begin{refsection}[citations/no_autocit/baca2019autonomous.bib]
  \nocite{*}
  \printbibliography[heading=none,title={},env=favoritebib]
\end{refsection}

\noindent
\fullcite{saska2017documentation}
\begin{refsection}[citations/no_autocit/saska2017documentation.bib]
\nocite{*}
\printbibliography[heading=none,title={},env=favoritebib]
\end{refsection}

\noindent
\fullcite{giernacki2019realtime}
\begin{refsection}[citations/no_autocit/giernacki2019realtime.bib]
  \nocite{*}
  \printbibliography[heading=none,title={},env=favoritebib]
\end{refsection}

\noindent
\fullcite{chudoba2014localization}
\begin{refsection}[citations/no_autocit/chudoba2014localization.bib]
  \nocite{*}
  \printbibliography[heading=none,title={},env=favoritebib]
\end{refsection}

\noindent
\fullcite{saikin2020wildfire}
\begin{refsection}[citations/no_autocit/saikin2020wildfire.bib]
  \nocite{*}
  \printbibliography[heading=none,title={},env=favoritebib]
\end{refsection}

\noindent
\fullcite{daniel2019inorbit}
\begin{refsection}[citations/no_autocit/daniel2019inorbit.bib]
  \nocite{*}
  \printbibliography[heading=none,title={},env=favoritebib]
\end{refsection}

\noindent
\fullcite{baca2018timepix}
\begin{refsection}[citations/no_autocit/baca2018timepix.bib]
  \nocite{*}
  \printbibliography[heading=none,title={},env=favoritebib]
\end{refsection}

\noindent
\fullcite{baca2018rospix}
\begin{refsection}[citations/no_autocit/baca2018rospix.bib]
  \nocite{*}
  \printbibliography[heading=none,title={},env=favoritebib]
\end{refsection}

\noindent
\fullcite{spurny2016complex}
\begin{refsection}[citations/no_autocit/spurny2016complex.bib]
  \nocite{*}
  \printbibliography[heading=none,title={},env=favoritebib]
\end{refsection}

\noindent
\fullcite{chudoba2016exploration}
\begin{refsection}[citations/no_autocit/chudoba2016exploration.bib]
  \nocite{*}
  \printbibliography[heading=none,title={},env=favoritebib]
\end{refsection}

\noindent
\fullcite{petrlik2020robust}
\begin{refsection}[citations/no_autocit/petrlik2020robust.bib]
  \nocite{*}
  \printbibliography[heading=none,title={},env=favoritebib]
\end{refsection}

%%}

\end{document}
