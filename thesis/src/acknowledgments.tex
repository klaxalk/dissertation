%!TEX root = ../main.tex

~\vfill{}

\section*{Acknowledgments}

Firstly, I would like to express my gratitude to my family for providing me with both material and mental support during my studies.
I am grateful that they allowed me to pursue the path of a student and a researcher, a career that is not known for its short-term benefits and securities.
Secondly, my thanks go to Martin Saska, my supervisor and colleague.
I am grateful for his trust he gave me during the founding and the growth of the MRS group.
I also do not take for granted the creative freedom I was given during my studies and work within the group.
Furthermore, my thanks go to all the present and past members of the MRS group.
The past six years was a \emph{bumpy ride} and I am grateful that I could make it with you.
Specifically, I would like to thank to my colleagues Vojtech Spurny, Daniel Hert, and Robert Penicka, who often \emph{shared the front seats} with me.
My own path would have been completely different without you.

My next thanks go to everyone who allowed me to work on the projects related to the radiation dosimetry for space applications.
Successful orchestration of research in the field of space instrumentation is comparably more difficult than research in mobile robotics.
Even though our results might be small in the grand scheme of things, the path towards them was no less difficult given the tight funding and limited know-how.
Among others, I am grateful to Vladimir Daniel (Czech Aerospace Research Institute), Adolf Inneman (Rigaku Innovative Technologies, s.r.o.), Jan Jakubek and Michal Platkevic (in that time at Institute of Experimental and Applied Physics, CTU) who made the VZLUSAT-1 nanosatellite project possible and successful.
Moreover, I am grateful to our colleagues at the University of Iowa and at the Pennsylvania State University for the opportunity to collaborate, despite the undoubtedly asymmetrical gains from the collaboration.
Specifically, I would like to thank to my colleagues and friends Randall Mc'Entaffer, Ted Schultz and James Tutt who went above and beyond with their hospitality during my visits at their laboratories.

During my Ph.D. studies, my work had been supported by the Czech Technical University in Prague through a Ph.D. scholarship and by the grants SGS15/157/OHK3/2T/13 and SGS17/187/OHK3/3T/13.
The Ministry of education of the Czech Republic supported the work by the grant no. 7AMB16FR017, and no. LH11053, and by OP VVV funded project CZ.02.1.01/0.0/0.0/16\_019/0000765 ``Research Center for Informatics''.
The Czech Science Foundation supported this work through projects no. 17-16900Y, no. 18-10088Y, and no. 20-10280S.
The Technology Agency of the Czech Republic supported this work through project no. FW01010317.
The European Union’s Horizon 2020 research and innovation programme supported this work under the grant agreement No 871479.
The National Grid Infrastructure MetaCentrum provided access to computing and storage facilities under the programme CESNET 569/2015, and LM2015042.
The Khalifa University of Science funded our participation in the MBZIRC 2017 and MBZIRC 2020 competitions that also motivated this work.
The work on the outer space radiation dosimetry and measurements would not be possible without the support of the Technology Agency of the Czech Republic projects no. TA03011329, no. TA04011295, the Czech Science Foundation projects no. GA13-33324S, GJ18-10088Y, and the project MSMT LH14039 of the Ministry of education youth and sports of the Czech Republic.
The work has been done on behalf of Medipix2 and Medipix3 collaborations.

\vspace{2.5cm}
