%!TEX root = ../main.tex

\section*{Acknowledgments}

Firstly, I would like to express my gratitude to my family for providing me with material and mental support during my studies.
I am grateful that they allowed me to pursue a student's and a researcher's path, a career that is not known for its immediate benefits and securities.
Secondly, my thanks go to Martin Saska, my supervisor and dear colleague.
Thank you for your guidance, leadership, and care that you provide to all your students.
I am grateful for trusting in me and I also do not take for granted the creative freedom I was given during my studies and work within the group.
Furthermore, I thank Michal Platkevic, my co-supervisor, for his guidance in the radiation dosimetry field.
I am also grateful to Tomas Krajnik for motivating me to apply to the CTU in Prague and later supervising me during my first steps in aerial robotics.
Finally, my thanks go to all the present and past members of the MRS group.
The past six years have been a \emph{bumpy ride}, and I am grateful that I could make it with you.
Specifically, I would like to thank my colleagues Vojtech Spurny, Daniel Hert, and Robert Penicka, who often \emph{shared the front seats} with me.
My path would have probably been different without you.
I am grateful to all other members of FEE, CTU in Prague, for creating a flourishing research environment where collaboration is encouraged.

My following thanks go to everyone who allowed me to work on the projects related to radiation dosimetry for space applications.
Successful orchestration of research in space instrumentation is comparably more difficult than research in mobile robotics.
Even though our results might be small in the grand scheme of things, the path towards them was no less difficult given the tight funding and limited know-how.
Among others, I am grateful to Vladimir Daniel (Czech Aerospace Research Institute), Adolf Inneman (Rigaku Innovative Technologies, s.r.o.), Jan Jakubek, and Michal Platkevic (at the time at the IEAP, CTU in Prague).
Without you, the VZLUSAT-1 nanosatellite would not have seen the light of day.
Moreover, I am grateful to our colleagues at the University of Iowa and Pennsylvania State University.
Thank you for the opportunity to collaborate, despite the probably asymmetrical gains from the collaboration.
I want to thank, among others, Randall Mc'Entaffer, Ted Schultz, and James Tutt for their hospitality during my visits to their laboratories.

During my Ph.D. studies, my work had been supported by the taxpayers of the Czech Republic through a Ph.D. scholarship.
My work had also been supported by the Czech Technical University grants SGS15/157/OHK3/2T/13 and SGS17/187/OHK3/3T/13.
The Ministry of education of the Czech Republic supported the work by grant no. 7AMB16FR017, and no. LH11053, and by OP VVV funded project CZ.02.1.01/0.0/0.0/16\_019/0000765 ``Research Center for Informatics''.
The Czech Science Foundation supported this work through projects no. 17-16900Y, no. 18-10088Y, and no. 20-10280S.
The Technology Agency of the Czech Republic supported this work through project no. FW01010317.
The European Union's Horizon 2020 research and innovation program supported this work under grant agreement No 871479.
The National Grid Infrastructure MetaCentrum provided access to computing and storage facilities under CESNET 569/2015 and LM2015042.
The Khalifa University of Science funded our MBZIRC 2017 and MBZIRC 2020 participation that also motivated this work.
The work on the outer space radiation dosimetry and measurements was supported by the Technology Agency of the Czech Republic projects no. TA03011329, no. TA04011295, the Czech Science Foundation projects no. GA13-33324S, GJ18-10088Y, and the project MSMT LH14039 of the Ministry of Education of the Czech Republic.
The work has been done on behalf of Medipix2 and Medipix3 collaborations.

\vspace{2.5cm}
