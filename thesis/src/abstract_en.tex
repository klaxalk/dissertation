%!TEX root = ../main.tex

\begin{changemargin}{0.8cm}{0.8cm}

~\vfill{}

\section*{Abstract}
\vskip 0.5em

The study of autonomous \acp{UAV} has become a prominent sub-field of mobile robotics.
Multirotor unmanned helicopters are systems useful on many levels of research.
Multicopters in research serve as a plant for testing new techniques for feedback control of dynamical systems, as sensor carriers for remote sensing applications, and as units of multi-robot systems.
A considerable amount of work is also being invested into research of principal sub-systems of multirotor \acp{UAV}.
Real-time localization, state estimation, modeling, feedback control, planning, and navigation are well-established and active research fields, each contributing to making \acp{UAV} autonomous, robust, and safe.
This thesis focuses on remote sensing with \acp{UAV} systems.
The first part of the thesis is dedicated to developing a novel \ac{UAV} control system, designed for real-world testing and evaluation of new methods.
The control and estimation system supports replicable research by allowing realistic simulations and real-world experiments.
The second part of the thesis is motivated by the challenges of the \acs{MBZIRC} 2017 and 2020 robotic competitions.
Specifically, we study the applications of groups of \acp{UAV} to fulfill the autonomous mission of collaborative localization and gathering of objects and their delivery to desired locations.
Initially (2017), only small metal disc-shaped objects are gathered by the \acp{UAV} into a large box.
Later (2020), the challenge was elevated into a task of autonomous brick wall construction by a group of \acp{UAV}.
Furthermore, we study the problem of concluding a \ac{UAV} mission by the autonomous landing of a \ac{UAV} on a moving ground vehicle.
All the challenges were tackled by developing a complete \ac{UAV} system and successfully performing an extensive experimental evaluation, that was finished by winning the international competitions.
The third part of the thesis focuses on a specific remote sensing field, the localization and mapping of ionizing radiation by \acp{UAV}.
The ongoing research investigates the use of hybrid pixel radiation sensor from the Timepix family onboard \acp{UAV} for directional and spatial localization of compact radiation sources.

\vskip 1em

{\bf Keywords} \Keywords

\vskip 2.5cm

\end{changemargin}
