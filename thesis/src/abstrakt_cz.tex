%!TEX root = ../main.tex

\begin{changemargin}{0.8cm}{0.8cm}

~\vfill{}

\section*{Abstrakt}
\vskip 0.5em

\sloppy
Výzkum na poli autonomních bezpilotních prostředků (UAV) se stal významným oborem mobilní robotiky.
Vícerotorové bezpilotní helikoptéry jsou užitečné na mnoha úrovních výzkumu.
Vícerotorové UAV slouží jako systémy pro testování nových technik v oboru zpětnovazebního řízení dynamických systémů, jako nosiče senzorů pro vzdálené měření a také jako součásti výzkumu multi-robotických systémů.
Značné úsilí je také věnované do výzkumu klíčových sub-systémů fungování vícerotorových UAV.
Systémy lokalizace, estimace stavu, modelování, zpětnovazební řízení, plánování pohybu, a autonomní navigace jsou již zavedené a aktivní pole výzkumu.
Každé z nich přispívá k bezpečné a robustní autonomii bezpilotních prostředků.
Tato práce se zabývá vzdáleným měřením pomocí autonomních bezpilotních systémů.
První část práce je věnovaná vývoji nové řídicí platformy pro vícerotorové UAV, která byla navržena za účelem testování a vyhodnocování nových metod pro UAV v reálném prostředí.
Tento systém pro řízení a odhadování stavů UAV umožňuje replikovatelný výzkum a poskytuje možnost realistických simulací a testování na UAV v reálném prostředí mimo laboratoř.
Druhá část této práce je motivována výzvami mezinárodních robotických soutěží MBZIRC 2017 a 2020.
Konkrétně zde představujeme aplikaci skupin autonomních UAV za účelem mísí pro kolaborativní sběr objektů a jejich dopravení na zadaná místa.
Nejprve (2017), jsme se zabývali sběrem malých kovových disků, které byly pomocí UAV autonomně dopraveny do sběrného boxu.
Poté (2020) byl tento úkol posunut na automatické stavění cihlové zdi pomocí skupiny UAV.
Dále jsme se zabývali problémem zakončování autonomních misí, konkrétně autonomním přistáním UAV na jedoucím vozidle.
Všechny tyto výzvy byly vyřešeny za vývoje kompletních bezpilotních systémů a jejich experimentální evaluace, která byla završena výhrou v obou mezinárodních soutěžích.
Třetí část této práce se zaměřuje na specifický podobor vzdáleného měření, a to na lokalizaci a mapování zdrojů ionizujícího záření pomocí UAV.
Probíhající výzkum se zabývá využitím hybridních pixelových senzorů radiace z rodiny detektorů Timepix.
V práci zkoumáme využití detektorů Timepix pro určování směrová a prostorové informace o kompaktních zdrojích ionizující radiace.

\vskip 1em

{\bf Klíčová slova} \KlicovaSlova

\vskip 2.5cm

\end{changemargin}
